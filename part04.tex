\documentclass[english,serif,mathserif,xcolor=pdftex,dvipsnames,table]{beamer}
\usetheme[informal]{s3it}
\usepackage{s3it}

\title[Objects]{%
  Everything is an object!
}
\author[S3IT]{%
  S3IT: Services and Support for Science IT, \\
  University of Zurich
}
\date{June~23--24, 2014}

\begin{document}

% title frame
\maketitle


\begin{frame}
  \frametitle{What's an \emph{object}?}
  \textbf{A Python object is a bundle of variables and functions.}

  \+
  What variable names and functions comprise an object is defined
  by the object's \emph{class}.

  \+
  From one class specification, many objects can be
  \emph{instanciated}.  Different instances can assign different
  values to the object variables.

  \+
  Variables and functions in an instance are collectively called
  \emph{instance attributes}; functions are also termed \emph{instance
    methods}.
\end{frame}


\begin{frame}[fragile]
  \frametitle{Example: the \texttt{datetime} object, I}
  \begin{columns}[c]
    \begin{column}{0.5\textwidth}
\begin{lstlisting}
>>> ~\HL{from datetime import date}~
>>> dt1 = date(2012, 9, 28)
>>> dt2 = date(2012, 10, 1)
\end{lstlisting}
    \end{column}
    \begin{column}{0.45\textwidth}
      \raggedleft
      Import the \texttt{date} class from the standard
      library module \texttt{datetime}
    \end{column}
  \end{columns}
\end{frame}


\begin{frame}[fragile]
  \frametitle{Example: the \texttt{datetime} object, II}
  \begin{columns}[c]
    \begin{column}{0.5\textwidth}
\begin{lstlisting}
>>> from datetime import date
>>> ~\HL{dt1 = date(2012, 9, 28)}~
>>> dt2 = date(2012, 10, 1)
\end{lstlisting}
    \end{column}
    \begin{column}{0.5\textwidth}
      \raggedleft
      To instanciate an object, call the class name like a
      function.
    \end{column}
  \end{columns}
\end{frame}


\begin{frame}[fragile]
  \frametitle{Example: the \texttt{datetime} object, III}
\begin{lstlisting}
>>> dir(dt1)
['__add__', '__class__', ~\ldots~, 'ctime', 'day',
'fromordinal', 'fromtimestamp', 'isocalendar',
'isoformat', 'isoweekday', 'max', 'min', 'month',
'replace', 'resolution', 'strftime', 'timetuple',
'today', 'toordinal', 'weekday', 'year']
\end{lstlisting}

  \+
  The \texttt{dir} function can list all objects attributes.

  \+
  Note there is no distinction between instance variables and
  methods!
\end{frame}


\begin{frame}[fragile]
  \frametitle{Example: the \texttt{datetime} object, IV}
  \begin{columns}[c]
    \begin{column}{0.5\textwidth}
\begin{lstlisting}
>>> dt1.day
28
>>> dt1.month
9
>>> dt1.year
2012
\end{lstlisting}
    \end{column}
    \begin{column}{0.5\textwidth}
      \raggedleft
      Access to object attributes is done by suffixing the
      instance name with the attribute name, separated by a dot
      ``\texttt{.}''.
    \end{column}
  \end{columns}
\end{frame}


\begin{frame}
  \frametitle{Objects \emph{vs} modules}

  Modules are also namespaces of variables and functions.

  \+
  The dot operator `\texttt{.}' is also used to access variables
  and functions from modules.  The \texttt{dir()} function is also
  used to list variables and functions from modules.

  \+
  But each module has \emph{one and only one} instance in a Python
  program.
\end{frame}


\begin{frame}[fragile]
  \frametitle{Example: the \texttt{datetime} object, V}
  \begin{columns}[c]
    \begin{column}[b]{0.5\textwidth}
\begin{lstlisting}
>>> dt1 = date(2012, 9, 28)
>>> dt2 = date(2012, 10, 1)

>>> dt1.day
~\HL{28}~
>>> dt2.day
~\HL{1}~
\end{lstlisting}
    \end{column}
    \begin{column}[b]{0.5\textwidth}
      \raggedleft
      The same attribute can have different
      values in different instances!
    \end{column}
  \end{columns}
\end{frame}


\begin{frame}[fragile]
  \frametitle{Instance methods}
\begin{lstlisting}[showstringspaces=false]
>>> dt1.isoformat()
'2012-09-28'
\end{lstlisting}

  \+
  Invoke an instance method just like any other function.
\end{frame}


\begin{frame}[fragile]
  \frametitle{Everything is an object!}

  The \texttt{dir} built-in function is used to list the attributes of an object.

  \begin{python}
>>> dir("hello!")
  \end{python}
\end{frame}


\begin{frame}[fragile]
  \frametitle{Everything is an object!}

  The \texttt{dir} built-in function is used to list the attributes of an object.

  \begin{python}
>>> ~\HL{dir("hello!")}~
['__add__', '__class__', '__contains__',
 '__delattr__', '__doc__', '__eq__',
 ~\ldots~
'strip', 'swapcase', 'title',
'translate', 'upper', 'zfill']
\end{python}

\+\ldots a string is an object!
\end{frame}


\begin{frame}[fragile]
  \frametitle{Everything is an object!}

  \begin{python}
>>> ~\HL{dir([1,2,3])}~
['__add__', '__class__', '__contains__',
~\ldots~
'append', 'count', 'extend',
'index', 'insert', 'pop',
'remove', 'reverse', 'sort']
\end{python}

\+\ldots a \texttt{list} is an object!
\end{frame}


\begin{frame}[fragile]
  \frametitle{Everything is an object!}
  Indeed, you can do:

  \+
  \begin{python}
>>> "hello world!".split()
['hello', 'world!']
\end{python}

\+
\begin{python}
>>> [1,1,2,3,5].count(1)
2
\end{python}
\end{frame}


\begin{frame}[fragile]
  \frametitle{Everything is an object!}

\begin{python}
>>> ~\HL{dir(1)}~
\end{python}

\end{frame}


\begin{frame}[fragile]
  \frametitle{Everything is an object!}

  \begin{python}
>>> ~\HL{dir(1)}~
['__abs__', '__add__', '__and__',
~\ldots~
'conjugate', 'denominator',
'imag', 'numerator', 'real']
\end{python}

\+\ldots an \texttt{int} is an object!

\+
\begin{python}
>>> (1).numerator
2
>>> (1).denominator
1
\end{python}


\end{frame}


\end{document}


%%% Local Variables:
%%% mode: latex
%%% TeX-master: t
%%% End:
