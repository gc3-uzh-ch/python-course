\documentclass[english,serif,mathserif,xcolor=pdftex,dvipsnames,table]{beamer}
\usepackage{gc3}

\title[OOP]{%
  Objects, modules and variable scoping rules
}
\author[GC3]{%
  GC3: Grid Computing Competence Center, \\
  University of Zurich
}
\date{Mar.~12--13, 2013}

\begin{document}

% title frame
\maketitle


\begin{frame}
  \frametitle{What's an \emph{object}?}
  \textbf{A Python object is a bundle of variables and functions.}

  \+
  What variable names and functions comprise an object is defined
  by the object's \emph{class}.

  \+
  From one class specification, many objects can be
  \emph{instanciated}.  Different instances can assign different
  values to the object variables.

  \+
  Variables and functions in an instance are collectively called
  \emph{instance attributes}; functions are also termed \emph{instance
    methods}.
\end{frame}


\begin{frame}[fragile]
  \frametitle{Example: the \texttt{datetime} object, I}
  \begin{columns}[c]
    \begin{column}{0.5\textwidth}
\begin{lstlisting}
>>> @\HL{from datetime import date}@
>>> dt1 = date(2012, 9, 28)
>>> dt2 = date(2012, 10, 1)
\end{lstlisting}
    \end{column}
    \begin{column}{0.45\textwidth}
      \raggedleft
      Import the \texttt{date} class from the standard
      library module \texttt{datetime}
    \end{column}
  \end{columns}
\end{frame}


\begin{frame}[fragile]
  \frametitle{Example: the \texttt{datetime} object, II}
  \begin{columns}[c]
    \begin{column}{0.5\textwidth}
\begin{lstlisting}
>>> from datetime import date
>>> @\HL{dt1 = date(2012, 9, 28)}@
>>> dt2 = date(2012, 10, 1)
\end{lstlisting}
    \end{column}
    \begin{column}{0.5\textwidth}
      \raggedleft
      To instanciate an object, call the class name like a
      function.
    \end{column}
  \end{columns}
\end{frame}


\begin{frame}[fragile]
  \frametitle{Example: the \texttt{datetime} object, III}
\begin{lstlisting}
>>> dir(dt1)
['__add__', '__class__', @\ldots@, 'ctime', 'day',
'fromordinal', 'fromtimestamp', 'isocalendar',
'isoformat', 'isoweekday', 'max', 'min', 'month',
'replace', 'resolution', 'strftime', 'timetuple',
'today', 'toordinal', 'weekday', 'year']
\end{lstlisting}

  \+
  The \texttt{dir} function can list all objects attributes.

  \+
  Note there is no distinction between instance variables and
  methods!
\end{frame}


\begin{frame}[fragile]
  \frametitle{Example: the \texttt{datetime} object, IV}
  \begin{columns}[c]
    \begin{column}{0.5\textwidth}
\begin{lstlisting}
>>> dt1.day
28
>>> dt1.month
9
>>> dt1.year
2012
\end{lstlisting}
    \end{column}
    \begin{column}{0.5\textwidth}
      \raggedleft
      Access to object attributes is done by suffixing the
      instance name with the attribute name, separated by a dot
      ``\texttt{.}''.
    \end{column}
  \end{columns}
\end{frame}


\begin{frame}
  \frametitle{Objects \emph{vs} modules}

  Modules are also namespaces of variables and functions.

  \+
  The dot operator `\texttt{.}' is also used to access variables
  and functions from modules.  The \texttt{dir()} function is also
  used to list variables and functions from modules.

  \+
  But each module has \emph{one and only one} instance in a Python
  program.
\end{frame}


\begin{frame}[fragile]
  \frametitle{Example: the \texttt{datetime} object, V}
  \begin{columns}[c]
    \begin{column}[b]{0.5\textwidth}
\begin{lstlisting}
>>> dt1 = date(2012, 9, 28)
>>> dt2 = date(2012, 10, 1)

>>> dt1.day
@\HL{28}@
>>> dt2.day
@\HL{1}@
\end{lstlisting}
    \end{column}
    \begin{column}[b]{0.5\textwidth}
      \raggedleft
      The same attribute can have different
      values in different instances!
    \end{column}
  \end{columns}
\end{frame}


\begin{frame}[fragile]
  \frametitle{Instance methods}
\begin{lstlisting}[showstringspaces=false]
>>> dt1.isoformat()
'2012-09-28'
\end{lstlisting}

  \+
  Invoke an instance method just like any other function.
\end{frame}


\begin{frame}
  \frametitle{No access control}
  There are no ``public''/``private''/etc. qualifiers for object
  attributes.

  \+
  \textbf{\emph{Any} code can create/read/overwrite/delete \emph{any} attribute on
    \emph{any} object.}

  \+
  There are \emph{conventions}, though:
  \begin{itemize}
  \item ``protected'' attributes: \texttt{\_name}
  \item ``private'' attributes: \texttt{\_\_name}
  \end{itemize}
  (But again, note that this is not \emph{enforced} by the system in
  any way.)

\end{frame}


\begin{frame}[fragile]
  \frametitle{Equality, identity}
  The \texttt{is} operator returns \texttt{True} if two names refer to
  the same instance; the \texttt{==} operator compares the
  \emph{values} of two objects.\footnote{A class can define how
    exactly the \texttt{==} operator should carry out the comparison.}

  \+
  Note that two instances may be equal in any respect yet be
  different instances: \emph{equality is not identity!}
\begin{lstlisting}
>>> dt4 = date(2012,9,28)
>>> dt5 = date(2012,9,28)
>>> dt4 == dt5
True
>>> dt4 is dt5
False
\end{lstlisting}
\end{frame}


\begin{frame}[fragile]
  \frametitle{Name resolution rules, I}
  \small

  Within a function body, names are resolved according to \href{http://stackoverflow.com/questions/291978/short-description-of-python-scoping-rules/292502#292502}{the LEGB rule}:
  \begin{description}
  \item[L] Local scope: any names defined in the current function;
  \item[E] Enclosing function scope: names defined in enclosing
    functions (outermost last);
  \item[G] global scope: names defined in the toplevel of the enclosing module;
  \item[B] Built-in names (i.e., Python's \texttt{\_\_builtins\_\_} module).
  \end{description}

  \+
  \textbf{Any name that is not in one of the above scopes \emph{must}
    be qualified.}

  \begin{references}
    \url{http://stackoverflow.com/questions/291978/short-description-of-python-scoping-rules/292502#292502}
  \end{references}
\end{frame}


\begin{frame}[fragile]
  \frametitle{Name resolution rules, II}
  \begin{columns}
    \begin{column}[t]{0.6\linewidth}
\begin{lstlisting}
import datetime as @\HL{dt}@

def today():
  td = dt.date.today()
  return "today is " + td.isoformat()

def hey(name):
  print("Hey " + name + "; " + today())

hey("you")
\end{lstlisting}
    \end{column}
    \begin{column}[t]{0.4\linewidth}
      \raggedleft
      Create a Python module and bind it to the
      \texttt{dt} name at global level.
    \end{column}
  \end{columns}
\end{frame}


\begin{frame}[fragile]
  \frametitle{Name resolution rules, III}
  \begin{columns}
    \begin{column}[t]{0.6\linewidth}
\begin{lstlisting}
import datetime as dt

def today():
  @\HL{td}@ = dt.date.today()
  return "today is " + @\HL{td}@.isoformat()

def hey(@\HL{name}@):
  print("Hey " + @\HL{name}@ + "; " + today())

hey("you")
\end{lstlisting}
    \end{column}
    \begin{column}[t]{0.4\linewidth}
      \raggedleft
      Unqualified name within a function: resolves to a local variable.
    \end{column}
  \end{columns}
\end{frame}


\begin{frame}[fragile]
  \frametitle{Name resolution rules, IV}
  \begin{columns}
    \begin{column}[t]{0.6\linewidth}
\begin{lstlisting}
import datetime as dt

def today():
  td = dt.date.today()
  return "today is " + td.isoformat()

def hey(name):
  print("Hey " + name + "; " + @\HL{today}@())

hey("you")
\end{lstlisting}
    \end{column}
    \begin{column}[t]{0.4\linewidth}
      \raggedleft
      Unqualified name: since there is no local variable by that name,
      it resolves to a module-level binding, i.e., to the
      \texttt{today} function defined above.
    \end{column}
  \end{columns}
\end{frame}


\begin{frame}[fragile]
  \frametitle{Name resolution rules, V}
  \begin{columns}
    \begin{column}[t]{0.6\linewidth}
\begin{lstlisting}
import datetime as dt

def today():
  td = @\HL{dt.date}@.today()
  return "today is " + td.isoformat()

def hey(name):
  print("Hey " + name + "; " + today())

hey("you")
\end{lstlisting}
    \end{column}
    \begin{column}[t]{0.4\linewidth}
      \raggedleft
      Qualified name: instructs Python to search the
      \texttt{date} attribute within the \texttt{dt} module.
    \end{column}
  \end{columns}
\end{frame}


\begin{frame}[fragile]
  \frametitle{Name resolution rules, VI}
  \begin{columns}
    \begin{column}[t]{0.6\linewidth}
\begin{lstlisting}
import datetime as dt

def today():
  td = dt.date.today()
  return "today is " + @\HL{td.isoformat}@()

def hey(name):
  print("Hey " + name + "; " + today())

hey("you")
\end{lstlisting}
    \end{column}
    \begin{column}[t]{0.4\linewidth}
      \raggedleft
      Qualified name: Python searches the \texttt{isoformat} attribute
      within the \texttt{td} object instance.
    \end{column}
  \end{columns}
\end{frame}


% \begin{frame}[fragile]
%   \frametitle{Name resolution rules, II}
%   \begin{columns}
%     \begin{column}[t]{0.6\linewidth}
% \begin{lstlisting}
% import datetime as dt

% def today():
%   td = dt.date.today()
%   return "today is " + td.isoformat()

% def hey(name):
%   print("Hey " + name + "; " + today())

% hey("you")
% \end{lstlisting}
%     \end{column}
%     \begin{column}[t]{0.4\linewidth}
%       \raggedleft

%     \end{column}
%   \end{columns}
% \end{frame}


\end{document}


%%% Local Variables:
%%% mode: latex
%%% TeX-master: t
%%% End:
