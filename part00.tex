\documentclass[english,serif,mathserif,xcolor=pdftex,dvipsnames,table]{beamer}
\usetheme[informal]{gc3}
\usepackage{gc3}

\title[Introduction]{%
  Introduction to the Python programming language
}
\author[S3IT]{%
  S3IT: Services and Support for Science IT, \\
  University of Zurich
}
\date{Mar.~19--20, 2014}

\begin{document}

% title frame
\maketitle

\begin{frame}
  \begin{center}
    {\Huge Welcome!}
  \end{center}
\end{frame}


\begin{frame}
  \frametitle{What is S3IT?}

  \begin{center}
    {\em ``A partner for data and \\ compute-intensive science''}

    \+
    \begin{description}
    \item[Enable] researchers and projects to run simulations and data analysis.
    \item[Develop] tools to integrate, automate and scale scientific use cases.
    \item[Provide] access to {\em innovative} infrastructures and technologies.
    \end{description}

    \+
    {\em \small{Want to know more ? }\url{http://www.s3it.uzh.ch}}
  \end{center}
\end{frame}


\begin{frame}
  \begin{center}
    {\Huge And what about you?}
  \end{center}
\end{frame}


\begin{frame}
  \frametitle{Prerequisites}
  This course assumes a basic experience with computer programming.

  \+
  Any language should do, as long as you are already familiar with
  the concepts of variables and functions.
\end{frame}


\begin{frame}
  \frametitle{Where to find the course material}

  These slides and all the example files can be downloaded from the
  course web page at:
  {\small\url{http://www.gc3.uzh.ch/edu/python}}

  \+
  Better keep a browser tab open on that page.

  \+
  (After the course is over, please rate it and the material using
  the feedback form on \href{http://www.gc3.uzh.ch/edu/python}{that
    same page})
\end{frame}

\begin{frame}
  \frametitle{A helpful tool}

  The \href{http://pythontutor.com}{Online Python Tutor} is a free
  tool to visualize the execution of small Python programs
  step-by-step.

  \+
  \href{http://tinyurl.com/cf5ftwr}{%
    \centering
    \includegraphics[width=1.0\linewidth,viewport=0 600 500 750,clip]{fig/pythontutor}
  }

  \+ Feel free to use it for the course exercises and your own code:
  \url{http://pythontutor.com/visualize.html}
\end{frame}

\begin{frame}
  \frametitle{Further reading}

  \begin{itemize}
    \item \textbf{The Python tutorial},
      {\small \url{http://docs.python.org/tutorial/}}
    \item {The Zen of Python in 3 days},
      {\small \url{http://pixelmonkey.org/pub/python-training/}}
    \item {Python for Java programmers},
      {\small \url{http://python4java.necaiseweb.org/Main/TableOfContents}}
  \end{itemize}

  \+   For an extensive and commented list, see:
  {\footnotesize\url{https://github.com/gc3-uzh-ch/python-course/blob/master/refs.mdwn}}

\end{frame}


\begin{frame}[fragile]
  \frametitle{Python 2 \emph{vs} Python 3}

  There are currently two major versions of Python available, with
  slightly different syntax and features.

  \+
  Python 2.7 is the last release in the 2.x series.

  \+
  Python 3.x has a more polished syntax, removing inconsistencies and
  some historical baggage.

  \+
  But Python 2.x is still the default on most Linux distributions
  and some major Python packages have not yet been ported to Py3, so
  \textbf{we shall focus on Py2 syntax}.

  \+
  {\footnotesize\em
    Watch a debate between ``Pro'' and ``Contra'' advocates:
    \url{http://www.physik.uzh.ch/~nchiapol/webm/3_1_Python3.webm}}

  \+
  {\footnotesize\em
    Explore the key differences:
    \url{http://tinyurl.com/py2-and-py3-key-differences}}
\end{frame}


\begin{frame}
  \frametitle{Next steps}

  The course will be structured as a mixture of slides and hands-on
  sessions for practicing Python programming.  The S3IT folks are here
  to help: ask them questions!

  \+
  So, the very first step is to set up your workstation so that you
  can edit files and run Python code.
\end{frame}



\end{document}

%%% Local Variables:
%%% mode: latex
%%% TeX-master: t
%%% End:
