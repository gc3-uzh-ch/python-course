\documentclass[english,serif,mathserif,xcolor=pdftex,dvipsnames,table]{beamer}
\usetheme[informal]{s3it}
\usepackage{s3it}

\title[Part 5]{%
  String manipulation and file I/O
}
\author[S3IT]{%
  S3IT: Services and Support for Science IT, \\
  University of Zurich
}
\date{June~23--24, 2014}

\begin{document}

% title frame
\maketitle

\begin{frame}[fragile]
  \frametitle{File I/O, I}

  \begin{describe}{\ttfamily stream = open(path,mode)}
    Return a Python \texttt{file} object for reading or writing the
    file located at \texttt{path}.  Mode is one of '\texttt{r}',
    '\texttt{w}' or '\texttt{a}' for reading, writing (truncates on open), appending.
    You can add a `\texttt{+}' character to enable read+write (other
    effects being the same).
  \end{describe}

  \begin{describe}{\ttfamily \emph{stream}.close()}
    Close an open file.
  \end{describe}

  \begin{describe}{\ttfamily \textbf{for} line \textbf{in} stream:}
    Loop over lines in the file one by one.
  \end{describe}

  \begin{references}
    \url{http://docs.python.org/library/stdtypes.html#file-objects}
  \end{references}
\end{frame}


\begin{frame}[fragile]
  \frametitle{File I/O, II}

  The \lstinline|read(n)| method can be used to read \emph{at most}
  \lstinline|n| bytes from a file-like object:
\begin{lstlisting}
>>> s = stream.read(2)
>>> print(s)
'py'
\end{lstlisting}
  If \lstinline|n| is omitted, \texttt{read()} reads until end-of-file.

  \begin{references}
    \url{http://docs.python.org/library/stdtypes.html#file-objects}
  \end{references}
\end{frame}


\begin{frame}[fragile]
  \begin{exercise}
    Write a function \lstinline|cat(filename)| that prints the whole contents of a file.

    \+
    Test it with the
    \href{https://raw.github.com/gc3-uzh-ch/python-course/master/welcome.py}{welcome.py}
    file:
\begin{lstlisting}
>>> cat('welcome.py')
#! /usr/bin/env python

print ("Welcome to Python!")
\end{lstlisting}
  \end{exercise}

  \+
  \begin{exercise}
    Write a function \lstinline|load_data(filename)| that reads a file
    containing one integer number per line, and return a list of the
    integer values.

    \+
    Test it with the
    \href{https://raw.github.com/gc3-uzh-ch/python-course/master/values.dat}{values.dat}
    file:
\begin{lstlisting}
>>> load_data('values.dat')
[299850, 299740, 299900, 300070, 299930]
\end{lstlisting}
  \end{exercise}
\end{frame}

\begin{frame}[fragile]
  \frametitle{List comprehensions, I}
  Python has a better and more compact syntax for \emph{filtering} elements
  of a list and/or \emph{applying} a function to them.

  \+
  The previous example:
  \begin{lstlisting}
data = []
for num in open('values.dat').readlines():
    data.append(int(num))
  \end{lstlisting}
  \+
  can be written using \textit{list comprehension}:
  \begin{python}
data = [ int(line) for line in open('values.dat') ]
  \end{python}

\end{frame}


\begin{frame}[fragile]
  \frametitle{List comprehensions, II}
  \def\e{\ttfamily\itshape}

  The general syntax of a list comprehension is:
  \begin{python}
    ~\bf[~ ~\e expr~ for ~\e var~ in ~\e iterable~ if ~\e condition~ ~\bf]~
  \end{python}
  where:
  \begin{description}
  \item[\e expr] is any Python expression;
  \item[\e iterable] is a (generalized) sequence;
  \item[\e condition] is a boolean expression, depending on
    {\e var};
  \item[\e var] is a variable that will be bound in turn to each item
    in {\e iterable} which satisfies {\e condition}.
  \end{description}

  \+ \textit{Create a new list, and for each \textbf{var} in the
    sequence \textbf{iterable}, if \textbf{condition} is true then add
    \textbf{expr} to the list.}
\end{frame}


\begin{frame}[fragile]
  \begin{exercise}
    Write a function called \texttt{load\_data2(filename, bound)}
    that, \textit{using comprehensions}, reads a file containing one
    integer number per line, and return a list of the integer values
    \textit{lesser than} \texttt{bound}.

    \+
    Test it with the
    \href{https://raw.github.com/gc3-uzh-ch/python-course/master/values.dat}{values.dat}
    file:
    \begin{python}
>>> load_data2('values.dat', 300000)
[299850, 299740, 299900, 299930]
    \end{python}

  \end{exercise}
\end{frame}


\begin{frame}[fragile]
  \frametitle{Operations on strings, I}
  \begin{describe}{%
      \lstinline|s.capitalize()|,
      \lstinline|s.lower()|,
      \lstinline|s.upper()|}
    Return a \emph{copy} of the string capitalized / turned all lowercase /
    turned all uppercase.
  \end{describe}

  \begin{describe}{\lstinline|s.split(t)|}
    Split \texttt{s} at every occurrence of \texttt{t} and return a list
    of parts.  If \texttt{t} is omitted, split on whitespace.
  \end{describe}

  \begin{describe}{\lstinline|s.startswith(t)|,
      \lstinline|s.endswith(t)|}
    Return \texttt{True} if \texttt{t} is the initial/final substring
    of \texttt{s}.
  \end{describe}

  \begin{references}
    \url{http://docs.python.org/library/stdtypes.html#string-methods}
  \end{references}
\end{frame}


\begin{frame}[fragile]
  \frametitle{Operations on strings, II}
  \begin{describe}{\lstinline|s.replace(old, new)|}
    Return a \emph{copy} of string \texttt{s} with all occurrences of
    substring \texttt{old} replaced by \texttt{new}.
  \end{describe}

  \begin{describe}{%
      \lstinline|s.lstrip()|,
      \lstinline|s.rstrip()|,
      \lstinline|s.strip()|}
    Return a \emph{copy} of the string with the leading (resp.\ trailing,
    resp.\ leading \emph{and} trailing) whitespace removed.
  \end{describe}

  \begin{references}
    \url{http://docs.python.org/library/stdtypes.html#string-methods}
  \end{references}
\end{frame}


\begin{frame}[fragile]
  \begin{exercise}
    Write a program that reads the
    \href{https://raw.github.com/gc3-uzh-ch/python-course/master/euro.csv}{euro.csv}
    file and populates a dictionary from it: currency names (first
    column) are the dictionary keys, conversion rates (second column)
    are the dictionary values.
  \end{exercise}

\end{frame}

\begin{frame}[fragile]
\begin{exercise}
    Write a function \lstinline|wordcount(filename)| that reads a text
    file and returns a dictionary, mapping words into occurrences
    (disregarding case) of that word in the text.  Test it with the
    \href{https://raw.github.com/gc3-uzh-ch/python-course/master/lorem_ipsum.txt}{lorem\_ipsum.txt} file:
    \begin{lstlisting}
>>> wordcount('lorem_ipsum.txt')
{'and': 3, 'model': 1, 'more-or-less': 1,
 'letters': 1, ...
    \end{lstlisting}

    \+ For the purposes of this
    exercise, a ``word'' is defined as a sequence of letters and the
    character ``-'', i.e., ``e-mail'' and ``more-or-less'' should both
    be counted as a single word.

    \+ You might want to have a look at the
    \href{http://docs.python.org/2/library/string.html}{string}
    module, for pre-defined sets of alphabetic and punctuation
    characters.
  \end{exercise}
\end{frame}


\begin{frame}[fragile]
  \frametitle{Filesystem operations, I}
  \small
  These functions are available from the \texttt{os} module.

  \begin{describe}{\lstinline|os.getcwd()|, \lstinline|os.chdir(path)|}
    Return the path to the current working directory /
    Change the current working directory to \texttt{path}.
  \end{describe}

  \begin{describe}{\lstinline|os.listdir(dir)|}
    Return list of entries in directory \texttt{dir} (omitting
    `\texttt{.}' and `\texttt{..}')
  \end{describe}

  \begin{describe}{\lstinline|os.mkdir(path)|}
    Create a directory; fails if the directory already exists.
    Assumes that all parent directories exist already.
  \end{describe}

  % \begin{describe}{\lstinline|os.makedirs(path)|}
  %   Create a directory; no-op if the directory already exists.
  %   Creates all the intermediate-level directories needed to contain
  %   the leaf.
  % \end{describe}

  \begin{describe}{\lstinline|os.rename(old,new)|}
    Rename a file or directory from \texttt{old} to \texttt{new}.
  \end{describe}

  \begin{references}
    \url{http://docs.python.org/library/os.html}
  \end{references}
\end{frame}


\begin{frame}[fragile]
  \frametitle{Filesystem operations, II}
  These functions are available from the \texttt{os.path} module.

  \begin{describe}{\lstinline|os.path.exists(path)|, \lstinline|os.path.isdir(path)|, \lstinline|os.path.isfile(path)|}
    Return \texttt{True} if \texttt{path} exists / is a directory / is
    a regular file.
  \end{describe}

  \begin{describe}{\lstinline|os.path.basename(path)|,
      \lstinline|os.path.dirname(path)|}
    Return the base name (the part after the last `\texttt{/}'
    character) or the directory name (the part before the last
    \texttt{/} character).
  \end{describe}

  \begin{describe}{\lstinline|os.path.abspath(path)|}
    Make \texttt{path} absolute (i.e., start with a \texttt{/}).
  \end{describe}

  \begin{references}
    \url{http://docs.python.org/library/os.path.html}
  \end{references}
\end{frame}


\begin{frame}[fragile]
  \frametitle{Command line arguments}
  The \texttt{sys} module provides access to some variables used or
  maintained by the interpreter.

  One of such variables is a list containing the arguments passed on
  the command line.

  \+
  \textbf{Example:} This is a simple script that
  prints the command line arguments used to invoke it:

  \begin{lstlisting}
import sys
print(sys.argv)
  \end{lstlisting}

\+
Calling the script as:
\begin{lstlisting}
$ python script.py foo bar
\end{lstlisting}
yields the following result:
\begin{lstlisting}
['script.py', 'foo', 'bar']
\end{lstlisting}
\end{frame}



\begin{frame}[fragile]
  \begin{exercise}\emph{(Homework)}
    Write a Python program \texttt{rename.py} with the following
    command-line:
\begin{lstlisting}[language=sh]
python rename.py EXT1 EXT2 DIR [DIR ...]
\end{lstlisting}
    where:
    \begin{description}
    \item[ext1,ext2] Are file name extensions (without the leading
      dot), e.g., \texttt{jpg} and \texttt{jpeg}.
    \item[dir] Is a directory path; possibly, many directories names can
      be given on the command-line.
    \end{description}

    The \texttt{rename.py} command should rename all files in
    directory DIR, that end with extension \texttt{ext1} to end with
    extension \texttt{ext2} instead.
  \end{exercise}
\end{frame}


\end{document}

%%% Local Variables:
%%% mode: latex
%%% TeX-master: t
%%% End:
