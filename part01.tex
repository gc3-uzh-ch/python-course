\documentclass[english,serif,mathserif,xcolor=pdftex,dvipsnames,table]{beamer}
\usetheme[informal]{gc3}
\usepackage{gc3}

\title[Set-up]{%
  Setting up your workspace for the course
}
\author[GC3]{%
  GC3: Grid Computing Competence Center, \\
  University of Zurich
}
\date{Mar.~12--13, 2013}

\begin{document}

% title frame
\maketitle


\begin{frame}
  \frametitle{Set-up for Python development}
  A bare-bones development environment consists of:
  \begin{itemize}
  \item A text editor (e.g.,
    \href{http://en.wikipedia.org/wiki/Gedit}{gedit},
    \href{http://hide1713.wordpress.com/2009/01/30/setup-perfect-python-environment-in-emacs/}{emacs},
    \href{http://blog.dispatched.ch/2009/05/24/vim-as-python-ide/}{vim})
  \item The Python interpreter (it is installed by default on
    Ubuntu and almost any other Linux distribution)
  \item A terminal application to run the interpreter in.
  \end{itemize}

  \+ See
  {\small \url{http://wiki.python.org/moin/IntegratedDevelopmentEnvironments}}
  for a commented list of {IDEs} with Python support.
\end{frame}


\begin{frame}[fragile]
  \frametitle{Ready, set, go!}

  Now that your development environment is set up, let's run your
  first Python program.
  \begin{enumerate}
  \item Create a directory for holding the course files.
  \item Download the \href{http://www.gc3.uzh.ch/welcome.py}{\texttt{welcome.py}} into it.
  \item Start the terminal application, and \texttt{cd} to the directory you created.
  \item Start the Python interpreter by typing the command \texttt{python}; when the interpreter is ready, it will display this prompt string `\texttt{>>>}'
  \item Now type the following command:
    \begin{lstlisting}
>>> import welcome
    \end{lstlisting}
  \item You should see a welcome message appear on your screen.
  \end{enumerate}

\end{frame}



\end{document}

%%% Local Variables:
%%% mode: latex
%%% TeX-master: t
%%% End:
