\documentclass[english,serif,mathserif,xcolor=pdftex,dvipsnames,table]{beamer}
\usepackage{gc3}

\title[Testing]{%
  Testing your code
}
\author[GC3]{%
  GC3: Grid Computing Competence Center, \\
  University of Zurich
}
\date{Mar.~12--13, 2013}

\begin{document}

% title frame
\maketitle


\begin{frame}
  \frametitle{Testing is \textbf{not} a waste of time.}
  A test is a piece of code that, when executed, check if the code you
  wrote behave as expected.

  \+
  Writing (and running often) tests\ldots
  \begin{itemize}
  \item helps you to easily find bugs.
  \item forces you to slow down and think.
  \item reduces the cost of change.
  \item reduces fear.
  \end{itemize}

  \+ \textbf{Test Driven Development:} writing tests \textit{before}
  writing the code helps you to write \textbf{better code.}
\end{frame}


\begin{frame}
  \frametitle{Different kind of tests} 

  Each test will check that a component of your code behaves in a
  specific way when feed it with a specific input.

  \+
  There are two kind of tests:
  \begin{itemize}
  \item Specific functions or class methods
    \begin{itemize}
    \item \textbf{unit tests}
    \end{itemize}
  \item bigger part of the program, or maybe the whole program.
    \begin{itemize}
    \item \textbf{functional tests}
    \end{itemize}
  \end{itemize}

  \+
  \pause
  yes, it's better to have both :-)
\end{frame}


\begin{frame}[fragile]
  \frametitle{Doctest (1/2)}

  The \lstinline|doctest| module searches for text that looks like
  interactive Python sessions in \textit{docstrings}, and then
  executes them to verify that they work exactly as shown.

  \+
  Combines \textit{documentation} and \textit{tests}
  \begin{lstlisting}
  def __mul__(self, scalar):
    """
    This works both as documentation *and* test.

      >>> v = Vector(1, 2)
      >>> print v * 2
      <2,4>
    """
  \end{lstlisting}
\end{frame}

\begin{frame}[fragile]
  \frametitle{Doctest (2/2)}

  Execute all the tests of your module with:

  \begin{lstlisting}[language=sh]
$ python -m doctest vector6.py
  \end{lstlisting}
%$

  \+
  The \lstinline|-m doctest| option tells python to \textit{execute} the
  \lstinline|doctest| module using your module file as argument.

  \+ \textbf{Exercise:} the file
  \href{http://www.gc3.uzh.ch/vector6.py}{\texttt{vector6.py}}
  has some doctest, but they are wrong. Find them and fix them!.
\end{frame}


\end{document}

%%% Local Variables:
%%% mode: latex
%%% TeX-master: t
%%% End:
