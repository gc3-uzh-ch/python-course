\documentclass[english,serif,mathserif,xcolor=pdftex,dvipsnames,table]{beamer}
 \usetheme[informal]{s3it}
\usepackage{s3it}

\title[Part 9]{%
  Exceptions
}
\author[S3IT]{%
  S3IT: Services and Support for Science IT, \\
  University of Zurich
}
\date{June~23--24, 2014}

\begin{document}

% title frame
\maketitle


\begin{frame}[fragile]
  \frametitle{Exceptions}

  ``Exceptions'' is the name given in Python to error conditions.

  \+
  You can write code that intercepts some error conditions and
  reacts.

%   Exceptions are objects that inherit from the built-in
%   \lstinline|Exception| class.

%   \+
%   To create a new exception just make a new class:
% \begin{lstlisting}
% class NewKindOfError(Exception):
%   """
%   Do use the docstring to document
%   what this error is about.
%   """
%   pass
% \end{lstlisting}

%   \+
%   Exceptions are handled by class name, so they usually do not need
%   any new methods (although you are free to define some if needed).

  \begin{seealso}
    \url{http://docs.python.org/library/exceptions.html}
  \end{seealso}
\end{frame}


\begin{frame}[fragile]
  \frametitle{What does an exception look like?}
\begin{lstlisting}
>>> stream.write('foo')
Traceback (most recent call last):
  File "<stdin>", line 1, in <module>
IOError: File not open for writing
\end{lstlisting}
\end{frame}


\begin{frame}[fragile]
  \frametitle{What does an exception look like?}
\begin{lstlisting}
>>> stream.write('foo')
Traceback (most recent call last):
  File "<stdin>", line 1, in <module>
IOError: ~\HL{File not open for writing}~
\end{lstlisting}

  \+
  This is the exception \emph{message}: it is supposed to be read
  by the (human) user.
\end{frame}


\begin{frame}[fragile]
  \frametitle{What does an exception look like?}
\begin{lstlisting}
>>> stream.write('foo')
Traceback (most recent call last):
  File "<stdin>", line 1, in <module>
~\HL{IOError}~: File not open for writing
\end{lstlisting}

  \+ This is the exception \emph{class name}; it is used for catching
  exceptions (syntax in the next slide).
\end{frame}


\begin{frame}[fragile]
\begin{lstlisting}
try:
  # code that might raise an exception
except SomeException:
  # handle some exception
except AnotherException, ex:
  # the actual Exception instance
  # is available as variable `ex'
finally:
  # performed on exit in any case
\end{lstlisting}

  \+
  The optional \lstinline|finally| clause is executed on exit from the
  \lstinline|try| or \lstinline|except| block in \emph{any} case.

  \begin{references}
    \scriptsize
    \url{http://docs.python.org/reference/compound_stmts.html#try}
\end{references}
\end{frame}


\begin{frame}[fragile]
  \frametitle{Raising exceptions}
  Use the \lstinline|raise| statement with an \texttt{Exception}
  instance:
\begin{lstlisting}
if an_error_occurred:
  raise AnError("Spider sense is tingling.")
\end{lstlisting}
\end{frame}


\begin{frame}[fragile]
  \frametitle{Raising exceptions}

  Use the \lstinline|raise| statement with an \texttt{Exception}
  instance:
\begin{lstlisting}
if an_error_occurred:
  raise ~\HL{AnError}~("Spider sense is tingling.")
\end{lstlisting}

  \+
  The exception class name specifies what kind of exception to raise.

  \+
  You must use an \emph{existing} exception class; we shall learn
  how to define new exceptions in the object-orientation lectures.
\end{frame}


\begin{frame}[fragile]
  \frametitle{Raising exceptions}

  Use the \lstinline|raise| statement with an \texttt{Exception}
  instance:
\begin{lstlisting}
if an_error_occurred:
  raise AnError(~\HL{"Spider sense is tingling."}~)
\end{lstlisting}

  \+
  The exception message is an arbitrary string.  It is meant for
  humans to read, so try to describe the error condition clearly and
  concisely.
\end{frame}


\begin{frame}[fragile]
  \frametitle{Raising exceptions}
  Within an \lstinline|except| clause, you can use \lstinline|raise|
  with no arguments to re-raise the current exception:
\begin{lstlisting}
try:
  something()
except ItDidntWork:
  do_cleanup()
  # re-raise exception to caller
  raise
\end{lstlisting}
\end{frame}


\begin{frame}[fragile]
  \begin{exercise}
    Enhance the function \lstinline|parse_data| defined in the ``Warm-up''
    Exercise: if an error is found in the data read from file,
    silently ignore it and drop the line.

    Test your code on the
    \href{https://raw.github.com/gc3-uzh-ch/python-course/master/rt.tsv}{\texttt{rt.dirty.csv}}
    data file: your code should detect and ignore the 4 malformed
    lines in the file.
  \end{exercise}
\end{frame}


\end{document}

%%% Local Variables:
%%% mode: latex
%%% TeX-master: t
%%% End:
