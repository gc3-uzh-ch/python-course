\documentclass[english,serif,mathserif,xcolor=pdftex,dvipsnames,table]{beamer}
\usepackage{gc3}

\title[OOP2]{%
  Object-oriented Python, II
}
\author[GC3]{%
  GC3: Grid Computing Competence Center, \\
  University of Zurich
}
\date{Sep.~28, 2012}

\begin{document}

% title frame
\maketitle


\begin{frame}[fragile]
  \begin{columns}[t]
    \begin{column}{0.5\textwidth}
\begin{lstlisting}
import unittest as ut
from minmax import MinMax

class TestMinMax(ut.TestCase):

    def test_init(self):
        m = MinMax()
        self.assertEqual(m.min, None)
        self.assertEqual(m.max, None)

    def test_send42(self):
        m = MinMax()
        m.send(0)
        m.send(42)
        self.assertEqual(m.min, 0)
        self.assertEqual(m.max, 42)
\end{lstlisting}
    \end{column}
    \begin{column}{0.5\textwidth}
      \raggedleft 
      This is called a ``test case''.

      \+ 
      Do you see anything unusual in this class definition?
    \end{column}
  \end{columns}
  
  {\small Source code available at:
    \url{http://www.gc3.uzh.ch/test_minmax.py}}
\end{frame}


\begin{frame}[fragile]
  \begin{columns}[t]
    \begin{column}{0.5\textwidth}
\begin{lstlisting}
import unittest as ut
from minmax import MinMax

class TestMinMax@\HL{(ut.TestCase)}@:

    def test_init(self):
        m = MinMax()
        self.assertEqual(m.min, None)
        self.assertEqual(m.max, None)

    def test_send42(self):
        m = MinMax()
        m.send(0)
        m.send(42)
        self.assertEqual(m.min, 0)
        self.assertEqual(m.max, 42)

if __name__ == "__main__":
  ut.main()
\end{lstlisting}
    \end{column}
    \begin{column}{0.5\textwidth}
      \raggedleft 
      This is \emph{not} ``\texttt{(object)}''!
    \end{column}
  \end{columns}
\end{frame}


\begin{frame}[fragile]
  \begin{columns}[t]
    \begin{column}{0.5\textwidth}
\begin{lstlisting}
import unittest as ut
from minmax import MinMax

class TestMinMax(ut.TestCase):

    def test_init(self):
        m = MinMax()
        @\HL{self.assertEqual}@(m.min, None)
        @\HL{self.assertEqual}@(m.max, None)

    def test_send42(self):
        m = MinMax()
        m.send(0)
        m.send(42)
        @\HL{self.assertEqual}@(m.min, 0)
        @\HL{self.assertEqual}@(m.max, 42)

if __name__ == "__main__":
  ut.main()
\end{lstlisting}
    \end{column}
    \begin{column}{0.5\textwidth}
      \raggedleft 
      \emph{Where have these functions been defined?}
    \end{column}
  \end{columns}
\end{frame}


\begin{frame}[fragile]
  Still the program runs fine!

  \+
\begin{lstlisting}[language=sh]
$ python test_minmax.py --verbose
test_init (__main__.TestMinMax) ... ok
test_send0 (__main__.TestMinMax) ... ok
test_send42 (__main__.TestMinMax) ... ok

----------------------------------------
Ran 3 tests in 0.000s

OK
\end{lstlisting}
\end{frame}

\begin{frame}[fragile]
  This class is defined as a \emph{descendant} of
  the \texttt{unittest.TestCase} class.
  
  \+ 
\begin{lstlisting}
class TestMinMax@\HL{(ut.TestCase)}@:

    def test_init(self):
        m = MinMax()
        self.assertEqual(m.min, None)
        self.assertEqual(m.max, None)
\end{lstlisting}

  \+
  This means that it \emph{inherits} all the attributes defined
  in the \emph{ancestor} class.
\end{frame}


\begin{frame}[fragile]
\begin{lstlisting}
class TestMinMax(ut.TestCase):

    def test_init(self):
        m = MinMax()
        @\HL{self.assertEqual}@(m.min, None)
        @\HL{self.assertEqual}@(m.max, None)
\end{lstlisting}

  \+ 
  In particular, the \texttt{assertEqual} method is defined in the
  parent class \texttt{unittest.TestCase}.
\end{frame}


\begin{frame}
  What happens if a descendant class redefines a method already
  defined in an ancestor class?

  \+
  \pause{\em
    The method in the descendant class \emph{overrides} the
    method in the ancestor class.
  }
\end{frame}


\begin{frame}
  What happens if a descendant class defines a \lstinline|__init__|
  method?

  \+ \pause{\em 
    The \lstinline|__init__| in the descendant class
    \emph{overrides} the method in the ancestor class.  So
    \lstinline|__init__| of the parent class(es) will not be called.
  }
\end{frame}


\begin{frame}[fragile]
  \frametitle{Constructor chaining}
  % \begin{flushright}
  %   \footnotesize%
  %   {\em ``Explicit is better than implicit''}
  %   --- T.~Peters, \href{http://www.python.org/dev/peps/pep-0020/}{The
  %     Zen of Python}
  % \end{flushright}

    When a class is instanciated, Python only calls the first
    constructor it can find in the
    \href{http://www.python.org/download/releases/2.3/mro/}{class inheritance call-chain}.

    \+ \textbf{If you need to call a superclass' constructor, you need
      to do it \emph{explicitly}:}
    \begin{python}
class Application(Task):
  def __init__(self, ...):
    # do Application-specific stuff here
    Task.__init__(self, ...)
    # some more Application-specific stuff
    \end{python}

    \+
    Calling a superclass constructor is optional, and
    it can happen anywhere in the \lstinline|__init__| method body.
\end{frame}


\begin{frame}[fragile]
  \frametitle{Multiple-inheritance}
  Python allows multiple inheritance.

  \+
  Just list all the parent classes:
  \begin{python}
class C(A,B):
  # class definition
  \end{python}
  
  \+
  With multiple inheritance, it is your responsibility to call all the
  needed superclass constructors.

  \+ 
  Python uses the
  \href{http://www.python.org/download/releases/2.3/mro/}{C3
    algorithm} to determine the call precedence in an inheritance
  chain.

  \+
  You can always query a class for its ``method resolution order'',
  via the \lstinline|__mro__| attribute:
  \begin{lstlisting}[basicstyle=\scriptsize\ttfamily]
>>> C.__mro__
(<class 'ex.C'>, <class 'ex.A'>, <class 'ex.B'>, <type 'object'>)
  \end{lstlisting}
\end{frame}


\begin{frame}
  \frametitle{Detour: Regular Expression objects}
  The \texttt{re} module in the standard library provides
  \emph{regular expression searching}, allowing you to match a string
  against a pattern.

  \+
  \begin{describe}{re.search(pattern, string)}
    If \texttt{pattern} is matched anywhere in \texttt{string}, return
    a \emph{match object}.  Otherwise, return \texttt{None}.
  \end{describe}

  \+
  \begin{describe}{\emph{match}.group(0)}
    The entire string matched by \texttt{pattern} in a search operation.
  \end{describe}

  \+
  \begin{references}
    \url{http://docs.python.org/library/re.html}
  \end{references}
\end{frame}


\begin{frame}[fragile]
  \begin{exercise}
    Define a \texttt{Grep} class:
    \begin{itemize}
    \item a \texttt{Grep} instance is constructed by giving a file name and a regular expression pattern, e.g., \lstinline|g = Grep(filename, pattern)|
    \item Each call to the \texttt{next()} method returns the next line in the file that matches the regular expression \texttt{pattern}.
    \end{itemize}

    Use the \texttt{.readline()} method of file objects!
  \end{exercise}

  \+
  \begin{exercise}
    Define a \texttt{GrepOnlyMatching} class, similar to \texttt{Grep}
    except that its \texttt{next()} method returns only the part of
    the line that matched the \texttt{pattern} expression.
  \end{exercise}

  \+
  \begin{exercise}
    Define a \texttt{GrepExactly} class, similar to \texttt{Grep}
    except that \texttt{pattern} is now a fixed string, and the
    \texttt{next()} method returns lines that \emph{contain} it.
  \end{exercise}
\end{frame}


\begin{frame}[fragile]
  \frametitle{The ``Template method'' pattern}
\begin{lstlisting}
class Grep(object):

    def __init__(self, filename, pattern):
        self._file = open(filename, 'r')
        self._pattern = pattern

    def match(self, line):
        return re.search(self._pattern, line)

    def result(self, match, line):
        return line

    def next(self):
        line = self._file.next()
        match = self.match(line)
        while not match:
            line = self._file.next()
            match = self.match(line)
        return self.result(match, line)
\end{lstlisting}
\end{frame}


\begin{frame}[fragile]
  \frametitle{The ``Template method'' pattern, I}
  \begin{columns}[t]
    \begin{column}{0.5\textwidth}
\begin{lstlisting}
class Grep(object):
  
  # parts omitted

  def next(self):
    line = self._file.next()
    match = @\HL{self.match}@(line)
    while not match:
        line = self._file.next()
        match = @\HL{self.match}@(line)
    return @\HL{self.result}@(match, line)
\end{lstlisting}
    \end{column}
    \begin{column}{0.5\textwidth}
      \raggedleft 
      These calls delegate the actual matching and
      extraction of the result from the line to instance methods.
    \end{column}
  \end{columns}
\end{frame}


\begin{frame}[fragile]
  \frametitle{The ``Template method'' pattern, II}
  \begin{columns}[t]
    \begin{column}{0.5\textwidth}
\begin{lstlisting}

class GrepOnlyMatching(Grep):
  @\HL{\textbf{def} result}@(self, match, line):
    return match.group(0)

class GrepExactly(Grep):
  @\HL{\textbf{def} match}@(self, line):
    return (self._pattern in line)
\end{lstlisting}
    \end{column}
    \begin{column}{0.5\textwidth}
      \raggedleft 
      So we need only re-define those methods in derived
      classes to implement a variant behavior.
    \end{column}
  \end{columns}
\end{frame}


\end{document}

%%% Local Variables: 
%%% mode: latex
%%% TeX-master: t
%%% End: 
