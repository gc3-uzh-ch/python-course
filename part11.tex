\documentclass[english,serif,mathserif,xcolor=pdftex,dvipsnames,table]{beamer}
 \usetheme[informal]{gc3}
\usepackage{gc3}

\title[Part 7]{%
  Exceptions
}
\author[GC3]{%
  GC3: Grid Computing Competence Center, \\
  University of Zurich
}
\date{Mar.~19--20, 2014}

\begin{document}

% title frame
\maketitle


\begin{frame}[fragile]
  \frametitle{Exceptions}

  Exceptions are objects that inherit from the built-in
  \lstinline|Exception| class.

  \+
  To create a new exception just make a new class:
\begin{lstlisting}
class NewKindOfError(Exception):
  """
  Do use the docstring to document
  what this error is about.
  """
  pass
\end{lstlisting}

  \+
  Exceptions are handled by class name, so they usually do not need
  any new methods (although you are free to define some if needed).

  \begin{seealso}
    \url{http://docs.python.org/library/exceptions.html}
  \end{seealso}
\end{frame}


\begin{frame}[fragile]
\begin{lstlisting}
try:
  # code that might raise an exception
except SomeException:
  # handle some exception
except AnotherException, ex:
  # the actual Exception instance
  # is available as variable `ex'
else:
  # performed on normal exit from `try'
finally:
  # performed on exit in any case
\end{lstlisting}

  \+
  The optional \lstinline|else| clause is executed if and when control flows off the
  \emph{end} of the \lstinline|try| clause.

  \+
  The optional \lstinline|finally| clause is executed on exit from the
  \lstinline|try| or \lstinline|except| block in \emph{any} case.

  \begin{references}
    \scriptsize
    \url{http://docs.python.org/reference/compound_stmts.html#try}
\end{references}
\end{frame}


\begin{frame}[fragile]
  \frametitle{Raising exceptions}

  Use the \lstinline|raise| statement with an \texttt{Exception}
  instance:
\begin{lstlisting}
if an_error_occurred:
  raise AnError("Spider sense is tingling.")
\end{lstlisting}

  \+
  Within an \lstinline|except| clause, you can use \lstinline|raise|
  with no arguments to re-raise the current exception:
\begin{lstlisting}
try:
  something()
except ItDidntWork:
  do_cleanup()
  # re-raise exception to caller
  raise
\end{lstlisting}
\end{frame}


\begin{frame}[fragile]
  \begin{exercise}
    Enhance the function \lstinline|load_data| defined in Exercise 5B:
    if an error is found (for instance a line which cannot be converted to
    an integer), silently ignore it.

    Test your code on the \texttt{test.csv} data file:
\begin{lstlisting}
>>> load_data('test.csv')
[1, 2, 3, 4]
\end{lstlisting}
  \end{exercise}

  % \begin{exercise}
  %   The \texttt{os.mkdir()} function raises a \texttt{OSError}
  %   exception if asked to create a directory that already exists.

  %   \+
  %   Write a \texttt{mkdir\_p(path)} function that creates a
  %   directory at \texttt{path}, but does nothing if the directory
  %   already exists.  Return \texttt{True} if the directory has been
  %   actually created, and \texttt{False} if nothing was changed on the
  %   file system.
  % \end{exercise}
\end{frame}


\begin{frame}[fragile]
  \frametitle{Exception handling example}

Read lines from a CSV file, ignoring those that do not have the
required number of fields.  If other errors occur, abort.
Close the file when done.
\begin{lstlisting}
job_state = { } # empty dict
try:
  csv_file = open('jobs.csv', 'r')
  for line in csv_file:
    line = line.strip() # remove trailing newline
    try:
      name, jobid, state = line.split(",")
    except ValueError:
      continue # ignore line
    job_state[jobid] = state
except IOError:
  raise # up to caller
finally:
  csv_file.close()
\end{lstlisting}
\end{frame}


\end{document}

%%% Local Variables:
%%% mode: latex
%%% TeX-master: t
%%% End:
