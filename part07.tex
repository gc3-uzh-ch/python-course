\documentclass[english,serif,mathserif,xcolor=pdftex,dvipsnames,table]{beamer}
\usetheme[informal]{gc3}
\usepackage{gc3}

\title[Special methods]{%
  Special methods
}
\author[GC3]{%
  GC3: Grid Computing Competence Center, \\
  University of Zurich
}
\date{Mar.~19--20, 2014}

\begin{document}

% title frame
\maketitle


\begin{frame}
  \frametitle{What we shall see in this part}

  How to define custom behavior for Python's standard operators and
  functions on user-defined objects.

  \+
  (Technically called ``operator overloading.'')
\end{frame}


\begin{frame}[fragile]
  \frametitle{Python's special methods}

  \textbf{Names that start and end with two underscores (e.g.,
    \lstinline|__init__|) have a special significance in Python.}

  \+ Most of them directly map to Python's operators (e.g.,
  ``\texttt{+}'').

  \+
  \begin{references}
    \url{http://docs.python.org/2/reference/datamodel.html#special-method-names}
  \end{references}
\end{frame}


\begin{frame}[fragile]
  \frametitle{First Example (1/2)}
  \begin{columns}[t]
    \begin{column}{0.5\textwidth}
\begin{lstlisting}
class Vector(object):
  """A 2D Vector."""
  def __init__(self, x, y):
    self.x = x
    self.y = y
  def add(self, other):
    return Vector(self.x+other.x,
                  self.y+other.y)
  def mul(self, scalar):
    return Vector(scalar*self.x, scalar*self.y)
  def ~\HL{\_\_str\_\_}~(self):
    return ("<%g,%g>" % (self.x, self.y))
\end{lstlisting}
    \end{column}
    \begin{column}{0.5\textwidth}
      \raggedleft
      Let us rename the \texttt{show} method to \texttt{\_\_str\_\_}.
    \end{column}
  \end{columns}

  \+
  {\scriptsize Source code available at:
    \url{https://raw.github.com/gc3-uzh-ch/python-course/master/vector2.py}}
\end{frame}


\begin{frame}[fragile]
  \frametitle{First Example (2/2)}
\begin{lstlisting}
>>> from vector2 import Vector
>>> v = Vector(0,1)
>>> print(v)
<0,1>
\end{lstlisting}

  \+
  Now Python's built-in \texttt{print} behaves like the \texttt{show} method did!

  \+ Actually, \texttt{print} uses Python's built-in function
  \texttt{str()} to convert an object to a string, and then prints
  this string.

  \+ {\bfseries By defining the \lstinline|__str__| method, we
    override the default behavior of Python's \lstinline|str()| for
    objects of class \texttt{Vector}.}
\end{frame}


\begin{frame}[fragile]
  \frametitle{Second Example: equality testing (1/3)}

  Can we test two instanced of class \texttt{Vector} for equality?
\begin{lstlisting}
>>> from vector2 import Vector
>>> v1 = Vector(0,1)
>>> v2 = Vector(0,1)
>>> v1 == v2
~\HL{False}~
\end{lstlisting}

  \+ Python does not know how to test if two user defined objects are
  equal.

  \+ \textbf{By default ``\texttt{==}'' behaves like the
    ``\texttt{is}'' operator on user-defined classes}, i.e., two
  user-defined objects are considered equal if and only if they are
  the same object.

  \+ \textbf{This can be changed by adding a \lstinline|__eq__| method.}
\end{frame}


\begin{frame}[fragile]
  \frametitle{Equality \emph{vs} identity}
  The \texttt{is} operator returns \texttt{True} if two names refer to
  the same instance; the \texttt{==} operator compares the
  \emph{values} of two objects.\footnote{A class can define how
    exactly the \texttt{==} operator should carry out the comparison.}

  \+
  Note that two instances may be equal in any respect yet be
  different instances: \emph{equality is not identity!}
\begin{python}
>>> dt4 = date(2012,9,28)
>>> dt5 = date(2012,9,28)
>>> dt4 == dt5
True
>>> dt4 is dt5
False
\end{python}
\end{frame}


\begin{frame}[fragile]
  \frametitle{Second Example: equality testing (2/3)}
  \begin{columns}[t]
    \begin{column}{0.5\textwidth}
\begin{lstlisting}
class Vector(object):
  """A 2D Vector."""
  def __init__(self, x, y):
    self.x = x
    self.y = y
  # (code omitted)
  def __str__(self):
    return ("<%g,%g>" % (self.x, self.y))
  ~\HL{\textbf{def} \_\_eq\_\_(\textbf{self}, other):}~
    return (self.x == other.x) and (self.y == other.y)
\end{lstlisting}
    \end{column}
    \begin{column}{0.5\textwidth}
      \raggedleft
      Let us add an \lstinline|__eq__|~method.
    \end{column}
  \end{columns}

  \+
  {\scriptsize Source code available at:
    \url{https://raw.github.com/gc3-uzh-ch/python-course/master/vector3.py}}
\end{frame}


\begin{frame}[fragile]
  \frametitle{Second Example: equality testing (3/3)}
\begin{lstlisting}
>>> from vector3 import Vector
>>> v1 = Vector(0,1)
>>> v2 = Vector(0,1)
>>> v1 == v2
True
\end{lstlisting}

  \+ {\bfseries By defining the \lstinline|__eq__| method, we
    define the behavior of Python's equality test \lstinline|==| for
    objects of class \texttt{Vector}.}
\end{frame}


\begin{frame}[fragile]
  \frametitle{3rd Example: vector addition (1/3)}
  \begin{columns}[t]
    \begin{column}{0.5\textwidth}
\begin{lstlisting}
class Vector(object):
  """A 2D Vector."""
  def __init__(self, x, y):
    self.x = x
    self.y = y
  def __str__(self):
    return ("<%g,%g>" % (self.x, self.y))
  def __eq__(self, other):
    return (self.x == other.x) and (self.y == other.y)
  ~\HL{\textbf{def} \_\_add\_\_(\textbf{self}, other):}~
    return Vector(self.x+other.x, self.y+other.y)
\end{lstlisting}
    \end{column}
    \begin{column}{0.5\textwidth}
      \raggedleft
      The \lstinline|__add__| special method defines the behavior of the
      ``\texttt{+}'' operator.

      \+
      Let's just rename \texttt{add} \\ to \texttt{\_\_add\_\_}.
    \end{column}
  \end{columns}

  \+
  {\scriptsize Source code available at:
    \url{https://raw.github.com/gc3-uzh-ch/python-course/master/vector3.py}}
\end{frame}


\begin{frame}[fragile]
  \frametitle{3rd Example: vector addition (2/3)}

  Now vector addition works with the usual ``\texttt{+}'' operator:
\begin{lstlisting}
>>> from vector4 import Vector
>>> v1 = Vector(1,0)
>>> v2 = Vector(0,1)
>>> print (v1 + v2)
<1,1>
\end{lstlisting}
\end{frame}


\begin{frame}[fragile]
  \frametitle{3rd Example: vector addition (3/3)}

  What if we add inhomogeneous objects, e.g., a vector and a number?
\begin{lstlisting}
>>> from vector4 import Vector
>>> v1 = Vector(1,0)
>>> print (v1 + 5.0)
Traceback (most recent call last):
  File "<stdin>", line 1, in <module>
  File "vector4.py", line 14, in __add__
    return Vector(self.x+other.x, self.y+other.y)
~\HL{AttributeError: 'float' object has no attribute 'x'}~
\end{lstlisting}

  \+ In this case, an error is the correct behavior: a vector can only
  be summed to another vector.

  \+ We shall see in the next part that sometimes it makes sense to
  allow inhomogenneous operations, and how to implement them.
\end{frame}

\begin{frame}[fragile]
  \frametitle{4th Example: vector multiplication (1/3)}
  \begin{columns}[t]
    \begin{column}{0.5\textwidth}
\begin{lstlisting}
class Vector(object):
  """A 2D Vector."""
  def __init__(self, x, y):
    self.x = x
    self.y = y
  def __str__(self):
    return ("<%g,%g>" % (self.x, self.y))
  def __eq__(self, other):
    return (self.x == other.x) and (self.y == other.y)
  def __add__(self, other):
    return Vector(self.x+other.x, self.y+other.y)
  ~\HL{\textbf{def} \_\_mul\_\_(\textbf{self}, scalar):}~
    return Vector(scalar*self.x, scalar*self.y)
\end{lstlisting}
    \end{column}
    \begin{column}{0.5\textwidth}
      \raggedleft
      The \lstinline|__mul__| special method defines the behavior of
      the ``\texttt{*}'' operator.
    \end{column}
  \end{columns}
\end{frame}

\begin{frame}[fragile]
  \frametitle{4th Example: vector multiplication (2/3)}
  Now vector multiplication works with the usual ``\texttt{*}'' operator:
\begin{lstlisting}
>>> from vector5 import Vector
>>> v1 = Vector(1,2)
>>> v1 * 3 == Vector(3, 6)
>>> print (v1 * 2)
<2,4>
\end{lstlisting}
\end{frame}


\begin{frame}[fragile]
  \frametitle{4th Example: vector multiplication (3/3)}
  \small

  Note that:
\begin{lstlisting}
>>> from vector5 import Vector
>>> v1 = Vector(1,2)
>>> 3 * v1
Traceback (most recent call last):
  File "<stdin>", line 1, in <module>
TypeError: unsupported operand type(s) for *: 'int' and 'Vector'
\end{lstlisting}

  \+ Order matters! Our \lstinline|__mul__| method requires a \lstinline|Vector| instance first, and a number second.

  \+ The operation with swapped operands is called \lstinline|__rmul__|, but treating this in detail would take us too far!

  \+ Take-home message: \textbf{Operations defined with special
    methods are not automatically commutative, transitive or any other
    property you normally associate with the operators \texttt{+},
    \texttt{*}, etc.}
\end{frame}

\end{document}

%%% Local Variables:
%%% mode: latex
%%% TeX-master: t
%%% End:
