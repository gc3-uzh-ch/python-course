\documentclass[english,serif,mathserif,xcolor=pdftex,dvipsnames,table]{beamer}
 \usepackage{gc3}
%\usepackage{lsci}

\title[Part 7]{%
  List comprehensions and generators
}
\author[GC3]{%
  GC3: Grid Computing Competence Center, \\
  University of Zurich
}
\date{Mar.~12--13, 2013}

\begin{document}

% title frame
\maketitle

\begin{frame}[fragile]
  \frametitle{An easy exercise}
  Assume you have a list of words, and you want to create a new list
  containing all these words converted in uppercase.

\pause
\+
A possible solution
\begin{python}
upper = [ ]
for word in wordlist:
  upper.append(word.upper())
\end{python}
\end{frame}


\begin{frame}
  \frametitle{Another easy exercise}
  A \emph{dotfile} is a file whose name starts with a dot character
  ``\texttt{.}''.

  \+
  How can you list all dotfiles in a given directory?

  \+
  (Recall that the Python library call for listing the entries in a
  directory is \texttt{os.listdir()})
\end{frame}


\begin{frame}[fragile]
  \frametitle{A very basic solution}
  Use a \lstinline|for| loop to accumulate the results into a list:
  \begin{python}
dotfiles = [ ]
for entry in os.listdir(path):
  if entry.startswith('.'):
    dotfiles.append(entry)
  \end{python}
\end{frame}


\begin{frame}[fragile]
  \frametitle{List comprehensions, I}
  Python has a better and more compact syntax for \emph{filtering} elements
  of a list and/or \emph{applying} a function to them:
  \begin{python}
dotfiles = [ entry for entry in dotfiles 
             if entry.startswith('.') ]
  \end{python}
\+
\begin{python}
upper = [ word.upper() for word in wordlist ]
\end{python}
  
  \+ 
  This is called a \emph{list comprehension}.
\end{frame}


\begin{frame}[fragile]
  \frametitle{List comprehensions, II}
  \def\e{\ttfamily\itshape}
  
  The general syntax of a list comprehension is:
  \begin{python}
    @\bf[@ @\e expr@ for @\e var@ in @\e iterable@ if @\e condition@ @\bf]@
  \end{python}
  where:
  \begin{description}
  \item[\e expr] is any Python expression;
  \item[\e iterable] is a (generalized) sequence;
  \item[\e condition] is a boolean expression, depending on
    {\e var};
  \item[\e var] is a variable that will be bound in turn to each item
    in {\e iterable} which satisfies {\e condition}.
  \end{description}

  \+
  The `{\lstinline|if| \e condition}' part is optional.
\end{frame}


% \begin{frame}[fragile]
%   \frametitle{Generator expressions}
%   \def\e{\ttfamily\itshape}

%   List comprehensions are a special case of \emph{generator expressions}:
%   \begin{python}
%     ( @\e expr@ for @\e var@ in @\e iterable@ if @\e condition@ )
%   \end{python}

%   \+
%   A generator expression is a valid iterable and can be used to
%   initialize tuples, sets, dicts, etc.:
%   \begin{python}
%     # the set of square numbers < 100
%     squares = set(n*n for n in range(10))
%   \end{python}

%   \+
%   Generator expressions are valid \emph{expression}, so they can be nested:
%   \begin{python}
%     # cartesian product of sets A and B
%     C = set( (a,b) for a in A for b in B )
%   \end{python}
% \end{frame}


% \begin{frame}[fragile]
%   \frametitle{Generators}
%   Generator expressions are a special case of \emph{generators}.

%   \+ A generator is like a function, except it uses \lstinline|yield|
%   instead of \lstinline|return|:
%   \begin{python}
%     def squares():
%       n = 0
%       while True:
%         yield n*n
%         n += 1
%   \end{python}

%   \+
%   At each iteration, execution resumes with the statement logically
%   following \lstinline|yield| in the generator's execution flow.

%   \+
%   There can be multiple \lstinline|yield| statements in a generator.

%   \begin{references}
%     \url{http://wiki.python.org/moin/Generators}
%   \end{references}
% \end{frame}

\end{document}

%%% Local Variables: 
%%% mode: latex
%%% TeX-master: t
%%% End: 
