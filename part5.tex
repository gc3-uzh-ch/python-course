\documentclass[english,serif,mathserif,xcolor=pdftex,dvipsnames,table]{beamer}
\usepackage{gc3}

\title[OOP]{%
  Object-oriented Python, I
}
\author[GC3]{%
  GC3: Grid Computing Competence Center, \\
  University of Zurich
}
\date{Sep.~28, 2012}

\begin{document}

% title frame
\maketitle


\begin{frame}
  \frametitle{What's an \emph{object}?}
  A Python object is a bundle of variables and functions.

  \+ 
  What variable names and functions comprise an object is defined
  by the object's \emph{class}.

  \+ 
  From one class specification, many objects can be
  \emph{instanciated}.  Different instances can assign different
  values to the object variables.
  
  \+ 
  Variables and functions in an instance are collectively called
  \emph{instance attributes}; functions are also termed \emph{instance
    methods}.
\end{frame}


\begin{frame}
  \frametitle{Objects \emph{vs} modules}

  Modules are also namespaces of variables and functions.

  \+ 
  But each module has \emph{one and only one} instance in a Python
  program.
\end{frame}


\begin{frame}[fragile]
  \frametitle{Example: the \texttt{datetime} object, I}
  \begin{columns}[c]
    \begin{column}{0.5\textwidth}
\begin{lstlisting}
>>> @\HL{from datetime import date}@
>>> dt1 = date(2012, 9, 28)
>>> dt2 = date(2012, 10, 1)
\end{lstlisting}
    \end{column}
    \begin{column}{0.45\textwidth}
      \raggedleft 
      Import the \texttt{date} class from the standard
      library module \texttt{datetime}
    \end{column}
  \end{columns}
\end{frame}


\begin{frame}[fragile]
  \frametitle{Example: the \texttt{datetime} object, II}
  \begin{columns}[c]
    \begin{column}{0.5\textwidth}
\begin{lstlisting}
>>> from datetime import date
>>> @\HL{dt1 = date(2012, 9, 28)}@
>>> dt2 = date(2012, 10, 1)
\end{lstlisting}
    \end{column}
    \begin{column}{0.5\textwidth}
      \raggedleft 
      To instanciate an object, call the class name like a
      function.
    \end{column}
  \end{columns}
\end{frame}


\begin{frame}[fragile]
  \frametitle{Example: the \texttt{datetime} object, III}
\begin{lstlisting}
>>> dir(dt1)
['__add__', '__class__', @\ldots@, 'ctime', 'day', 
'fromordinal', 'fromtimestamp', 'isocalendar', 
'isoformat', 'isoweekday', 'max', 'min', 'month', 
'replace', 'resolution', 'strftime', 'timetuple', 
'today', 'toordinal', 'weekday', 'year']
\end{lstlisting}

  \+
  The \texttt{dir} function can list all objects attributes.

  \+ 
  Note there is no distinction between instance variables and
  methods!
\end{frame}


\begin{frame}[fragile]
  \frametitle{Example: the \texttt{datetime} object, IV}
  \begin{columns}[c]
    \begin{column}{0.5\textwidth}
\begin{lstlisting}
>>> dt1.day
28
>>> dt1.month
9
>>> dt1.year
2012
\end{lstlisting}
    \end{column}
    \begin{column}{0.5\textwidth}
      \raggedleft 
      Access to object attributes is done by suffixing the
      instance name with the attribute name, separated by a dot
      ``\texttt{.}''.
    \end{column}
  \end{columns}
\end{frame}


\begin{frame}[fragile]
  \frametitle{Example: the \texttt{datetime} object, V}
  \begin{columns}[c]
    \begin{column}{0.5\textwidth}
\begin{lstlisting}
>>> dt1.day
@\HL{28}@
>>> dt2.day
@\HL{1}@
\end{lstlisting}
    \end{column}
    \begin{column}{0.5\textwidth}
      \raggedleft 
      The same attribute can have different 
      values in different instances!
    \end{column}
  \end{columns}
\end{frame}


\begin{frame}
  \frametitle{No access control}
  There are no ``public''/``private''/etc. qualifiers for object
  attributes.

  \+
  \textbf{\emph{Any} code can create/read/overwrite/delete \emph{any} attribute on
    \emph{any} object.}

  \+
  There are \emph{conventions}, though:
  \begin{itemize}
  \item ``protected'' attributes: \texttt{\_name}
  \item ``private'' attributes: \texttt{\_\_name}
  \end{itemize}
  (But again, note that this is not \emph{enforced} by the system in
  any way.)

\end{frame}


\begin{frame}[fragile,fragile]
  \frametitle{Equality, identity}
  The \texttt{is} operator returns \texttt{True} if two names refer to
  the same instance; the \texttt{==} operator compares the
  \emph{values} of two objects.\footnote{A class can define how
    exactly the \texttt{==} operator should carry out the comparison.}

  \+
  Note that two instances may be equal in any respect yet be
  different instances: \emph{equality is not identity!}
\begin{lstlisting}
>>> dt4 = date(2012,9,28)
>>> dt5 = date(2012,9,28)
>>> dt4 == dt5
True
>>> dt4 is dt5
False
\end{lstlisting}
\end{frame}


\begin{frame}[fragile]
  \frametitle{Instance methods}
\begin{lstlisting}[showstringspaces=false]
>>> dt1.strftime('%a %d %b')
'Fri 28 Sep'
\end{lstlisting}

  \+ 
  Invoke an instance method just like any other function.
\end{frame}




\begin{frame}[fragile]
  \frametitle{User-defined classes, I}
  \begin{columns}[t]
    \begin{column}{0.5\textwidth}
\begin{lstlisting}
@\HL{\textbf{class} MinMax(object):}@

  def __init__(self):
    self.min = None
    self.max = None

  def send(self, val):
    if (self.min is None) or (val < self.min):
        self.min = val
    if (self.max is None) or (val > self.max):
        self.max = val
\end{lstlisting}
    \end{column}
    \begin{column}{0.5\textwidth}
      \raggedleft 
      A class definition starts with the keyword \texttt{class}.

      The class definition is indented relative to the \texttt{class}
      statement.
    \end{column}
  \end{columns}

  \+
  {\scriptsize Source code available at:
    \url{http://www.gc3.uzh.ch/minmax.py}}
\end{frame}


\begin{frame}[fragile]
  \frametitle{User-defined classes, II}
  \begin{columns}[t]
    \begin{column}{0.5\textwidth}
\begin{lstlisting}
class MinMax@\HL{(object)}@:

  def __init__(self):
    self.min = None
    self.max = None

  def send(self, val):
    if (self.min is None) or (val < self.min):
        self.min = val
    if (self.max is None) or (val > self.max):
        self.max = val
\end{lstlisting}
    \end{column}
    \begin{column}{0.5\textwidth}
      \raggedleft 
      This identifies user-defined~classes.

      \+
      (Do not leave it out or you'll get an ``old-style'' class, which
      is deprecated behavior.)
    \end{column}
  \end{columns}
\end{frame}


\begin{frame}[fragile]
  \frametitle{User-defined classes, III}
  \begin{columns}[t]
    \begin{column}{0.5\textwidth}
\begin{lstlisting}
class MinMax(object):

  @\HL{\textbf{def} \_\_init\_\_(\textbf{self}):}@
    self.min = None
    self.max = None

  def send(self, val):
    if (self.min is None) or (val < self.min):
        self.min = val
    if (self.max is None) or (val > self.max):
        self.max = val
\end{lstlisting}
    \end{column}
    \begin{column}{0.5\textwidth}
      \raggedleft 
      The {\bf def} keyword introduces a method~definition.

      \+
      Every method \emph{must} have at~least one argument,
      named~{\bf self}.
    \end{column}
  \end{columns}
\end{frame}


\begin{frame}[fragile]
  \frametitle{The \texttt{self} argument}

  \textbf{Every method of a Python object always has \texttt{self}
    as first argument.}

  \+
  However, you do not specify it when calling a method: it's
  automatically inserted by Python:
\begin{lstlisting}
>>> class ShowSelf(object):
...   def show(self):
...     print(self)
... 
>>> x = ShowSelf() # construct instance
>>> x.show() # `self' automatically inserted!
<__main__.ShowSelf object at 0x299e150>
\end{lstlisting}

  \+ 
  The \texttt{self} name is a reference to the object instance
  itself.  You \emph{need to} use \texttt{self} when accessing methods
  or attributes of this instance.
\end{frame}


\begin{frame}[fragile]
  \frametitle{Name resolution rules}
  \small

  Within a function body, names are resolved according to \href{http://stackoverflow.com/questions/291978/short-description-of-python-scoping-rules/292502#292502}{the LEGB rule}:
  \begin{description}
  \item[L] Local scope: any names defined in the current function;
  \item[E] Enclosing function scope: names defined in enclosing
    functions (outermost last);
  \item[G] global scope: names defined in the toplevel of the enclosing module;
  \item[B] Built-in names (i.e., Python's \texttt{\_\_builtins\_\_} module).
  \end{description}

  \+
  \textbf{Any name that is not in one of the above scopes \emph{must}
    be qualified.}

  \+
  So you have to write \texttt{self.min} to reference an attribute in
  this instance, \texttt{datetime.date} to mean a class defined in module
  \texttt{date}, etc.

  % \begin{references}
  %   \url{http://stackoverflow.com/questions/291978/short-description-of-python-scoping-rules/292502#292502}}
  % \end{references}
\end{frame}


\begin{frame}[fragile]
  \frametitle{Object initialization}
  \begin{columns}[t]
    \begin{column}{0.5\textwidth}
\begin{lstlisting}
class MinMax(object):

  @\HL{\textbf{def} \_\_init\_\_(\textbf{self}):}@
    self.min = None
    self.max = None

  def send(self, val):
    if (self.min is None) or (val < self.min):
        self.min = val
    if (self.max is None) or (val > self.max):
        self.max = val
\end{lstlisting}
    \end{column}
    \begin{column}{0.5\textwidth}
      \raggedleft
      The \lstinline|__init__| method has a special
      meaning: it is called when an instance is created.
    \end{column}
  \end{columns}
\end{frame}



\begin{frame}[fragile]
  \frametitle{Constructors}

  The \lstinline|__init__| method is the object constructor.
  It should \emph{never} return any value.

  \+ 
  You never call \lstinline|__init__| directly, it is invoked by
  Python when a new object is created from the class:
\begin{lstlisting}
# calls MinMax.\_\_init\_\_
m = MinMax()
\end{lstlisting}

  \+ 
  The arguments to \lstinline|__init__| are the arguments you
  should supply when creating a class instance.

  \+ 
  (Again, minus the \texttt{self} part which is automatically
  inserted by Python.)
\end{frame}


% \begin{frame}
%   \frametitle{No overloading}

%   \textbf{Python does not allow overloading of functions.}

%   \+
%   Any function.

%   \+
%   Hence, no overloading of constructors.

%   \+
%   So: \textbf{a class has one and only one constructor.}
% \end{frame}


\begin{frame}[fragile]
  \begin{exercise}
    Change the \texttt{MinMax} class so that it can be initialized
    with a sequence; the \texttt{.min} and \texttt{.max} attributes
    should be initialized to the minimum and maximum of that sequence.
    Example:
\begin{lstlisting}
>>> m = MinMax([1,2,3])
>>> m.min == 1
True
>>> m.max == 3
True
\end{lstlisting}
  \end{exercise}

  \begin{exercise}
    Augment the \texttt{MinMax} class so that it computes the
    average of the numbers given to it via the \texttt{send} method.
    Store this average in the \texttt{.average} instance attribute.
    Call the new class \texttt{MinMaxAvg}.
  \end{exercise}
\end{frame}


% \begin{frame}
%   \frametitle{Protocols}
%   A \emph{protocol}, in the Python jargon, is an informal
%   specification of what methods an object should implement in order to
%   be used in a certain context.

%   \+ For example, objects that implement the same methods as the
%   \texttt{file} object are called ``file-like'' and can be used in
%   stead of a real file.

%   \+ 
%   (This corresponds to the notion of \emph{interfaces} in other
%   programming languages, but interfaces are usually a formal language
%   construct.)
% \end{frame}


% \begin{frame}[fragile]
%   \frametitle{The Iterator Protocol}

%   An \emph{iterator} is an object that can be used in the
%   \texttt{in}-clause of a \texttt{for \ldots in \ldots} statement.

%   \+
%   An object can function as an iterator iff it implements a
%   \texttt{next()} method, that:
%   \begin{description}
%   \item[\emph{either}] returns the next value in the iteration,
%   \item[\emph{or}] use \lstinline|raise StopIteration| to signal the
%     end of the iteration.
%   \end{description}

%   \+
%   An object can be iterated over with \lstinline|for| if it implements a
%   \lstinline|__iter__()| method that returns an iterator.

%   \begin{references}
%     \url{http://www.python.org/dev/peps/pep-0234/}
%   \end{references}
% \end{frame}


% \begin{frame}[fragile]
%   \begin{columns}[t]
%     \begin{column}{0.5\textwidth}
% \begin{lstlisting}
% class Evens(object):

%   def __init__(self, numbers):
%     self._numbers = numbers

%   def next(self):
%       result = self._numbers.pop(0)
%       while (result % 2) != 0:
%           result = self._numbers.pop(0)
%       return result

%   def __iter__(self):
%       return self
% \end{lstlisting}
%     \end{column}
%     \begin{column}{0.5\textwidth}
%       \raggedleft
%       Iterate over the even numbers in a sequence.
%     \end{column}
%   \end{columns}

%   \+
%   {\scriptsize Source code available at:
%     \url{http://www.gc3.uzh.ch/evens.py}}
% \end{frame}


% \begin{frame}[fragile]
%   \frametitle{Example using \texttt{Evens}}
% \begin{lstlisting}
% >>> from evens import Evens
% >>> e = Evens([1,2,3,5,7,8])
% >>> e.next()
% 2
% >>> e.next()
% 8
% >>> e.next()
% Traceback (most recent call last):
%   File "<stdin>", line 1, in <module>
%   File "evens.py", line 12, in next
%     result = self._numbers.next()
% StopIteration
% \end{lstlisting}
% \end{frame}


% \begin{frame}[fragile]
%   \frametitle{Example using \texttt{Evens}}
% \begin{lstlisting}
% >>> from evens import Evens
% >>> e = Evens([1,2,3,5,7,8])
% >>> for num in e:
% ...   print num
% ... 
% 2
% 8
% \end{lstlisting}
% \end{frame}


% \begin{frame}
%   \begin{exercise}
%     Write a \texttt{GrepExactly} class that:
%     \begin{itemize}
%     \item is constructed by passing a file name and a string: 
%       \lstinline|ge = GrepExactly(filename, pattern)|
%     \item implements the iterator protocol
%     \item each call to its \texttt{next()} method returns the next line
%       in the file that contains the pattern string.
%     \end{itemize}
%   \end{exercise}
% \end{frame}

\end{document}

%%% Local Variables: 
%%% mode: latex
%%% TeX-master: t
%%% End: 
