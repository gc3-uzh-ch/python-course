\documentclass[english,serif,mathserif,xcolor=pdftex,dvipsnames,table]{beamer}
\usepackage{gc3}

\title[Part 4]{%
  String manipulation and file I/O
}
\author[R. Murri]{%
  \textbf{Riccardo Murri} \\
  GC3: Grid Computing Competence Center, \\
  University of Zurich
}
\date{Sep.~28, 2012}

\begin{document}

% title frame
\maketitle

\begin{frame}[fragile]
  \frametitle{File I/O}

  \begin{describe}{\ttfamily stream = open(path,mode)}
    Return a Python \texttt{file} object for reading or writing the
    file located at \texttt{path}.  Mode is one of '\texttt{r}',
    '\texttt{w}' or '\texttt{a}' for reading, writing (truncates on open), appending.
    You can add a `\texttt{+}' character to enable read+write (other
    effects being the same).
  \end{describe}

  \begin{describe}{\ttfamily \emph{stream}.close()}
    Close an open file.
  \end{describe}
  
  \begin{describe}{\ttfamily \textbf{for} line \textbf{in} stream:}
    Loop over lines in the file one by one.
  \end{describe}

  \begin{references}
    \url{http://docs.python.org/library/stdtypes.html#file-objects}
  \end{references}
\end{frame}


\begin{frame}[fragile]
  \frametitle{Strings to streams and back}

  The \texttt{StringIO} module creates a file-like object from a
  string:
\begin{lstlisting}
>>> from StringIO import StringIO
>>> # create a file-like object
>>> stream = StringIO("python")
\end{lstlisting}

  \+
  The \lstinline|read(n)| method can be used to read \emph{at most}
  \lstinline|n| bytes from a file-like object:
\begin{lstlisting}
>>> s = stream.read(2)
>>> s == 'py'
True
\end{lstlisting}
  If \lstinline|n| is omitted, \texttt{read()} reads until end-of-file.

  \begin{references}
    \url{http://docs.python.org/library/stringio.html}
  \end{references}
\end{frame}


\begin{frame}[fragile]
  \begin{exercise}
    Implement a function \lstinline|count_lines(stream)|, that returns
    the number of lines in \lstinline|stream| (a file-like object).
  \end{exercise}

  \+
  \begin{exercise}
    Write a function \lstinline|load_data(filename)| that reads a file
    containing one integer number per line, and return a list of the
    integer values.

    \+ 
    Test it with the
    \href{http://www.gc3.uzh.ch/values.dat}{values.dat}
    file: 
\begin{lstlisting}
>>> load_data('values.dat')
[299850, 299740, 299900, 300070, 299930]
\end{lstlisting}
  \end{exercise}
\end{frame}


\begin{frame}[fragile]
  \frametitle{Operations on strings, I}
  \begin{describe}{%
      \lstinline|s.capitalize()|,
      \lstinline|s.lower()|, 
      \lstinline|s.upper()|}
    Return a \emph{copy} of the string capitalized / turned all lowercase /
    turned all uppercase.
  \end{describe}
  
  \begin{describe}{\lstinline|s.split(t)|}
    Split \texttt{s} at every occurrence of \texttt{t} and return a list
    of parts.  If \texttt{t} is omitted, split on whitespace.
  \end{describe}
  
  \begin{describe}{\lstinline|s.startswith(t)|, 
      \lstinline|s.endswith(t)|}
    Return \texttt{True} if \texttt{t} is the initial/final substring
    of \texttt{s}.
  \end{describe}
  
  \begin{references}
    \url{http://docs.python.org/library/stdtypes.html#string-methods}
  \end{references}
\end{frame}


\begin{frame}[fragile]
  \frametitle{Operations on strings, II}
  \begin{describe}{\lstinline|S.replace(old, new)|}
    Return a \emph{copy} of string \texttt{S} with all occurrences of
    substring \texttt{old} replaced by \texttt{new}.
  \end{describe}
  
  \begin{describe}{%
      \lstinline|s.lstrip()|,
      \lstinline|s.rstrip()|, 
      \lstinline|s.strip()|}
    Return a \emph{copy} of the string with the leading / trailing /
    leading \emph{and} trailing whitespace removed.
  \end{describe}
  
  \begin{references}
    \url{http://docs.python.org/library/stdtypes.html#string-methods}
  \end{references}
\end{frame}


\begin{frame}[fragile]
  \begin{exercise}
    Write a program that reads the \href{http://www.gc3.uzh.ch/euro.csv}{euro.csv} file, and populates a dictionary from it: currency names (first column) are the dictionary keys, conversion rates (second column) are the dictionary values.
  \end{exercise}

  \+
\begin{exercise}
  Building upon the previous exercise, create a \lstinline|rates[][]|
  2D array that stores the convertion rate of two currencies given the
  name, e.g., \lstinline|rate['ITL']['DEM']| gives the conversion rate
  of Italian Liras to Deutsche Marks.
  \end{exercise}
\end{frame}


\begin{frame}[fragile]
  \frametitle{Filesystem operations, I}
  \small
  These functions are available from the \texttt{os} module.

  \begin{describe}{\lstinline|os.getcwd()|, \lstinline|os.chdir(path)|}
    Return the path to the current working directory / 
    Change the current working directory to \texttt{path}.
  \end{describe}
    
  \begin{describe}{\lstinline|os.listdir(dir)|}
    Return list of entries in directory \texttt{dir} (omitting
    `\texttt{.}' and `\texttt{..}')
  \end{describe}
  
  \begin{describe}{\lstinline|os.mkdir(path)|}
    Create a directory; fails if the directory already exists.
    Assumes that all parent directories exist already.
  \end{describe}
  
  \begin{describe}{\lstinline|os.makedirs(path)|}
    Create a directory; no-op if the directory already exists.
    Creates all the intermediate-level directories needed to contain
    the leaf.
  \end{describe}
  
  \begin{references}
    \url{http://docs.python.org/library/os.html}
  \end{references}
\end{frame}


\begin{frame}[fragile]
  \frametitle{Filesystem operations, II}
  These functions are available from the \texttt{os.path} module.

  \begin{describe}{\lstinline|os.path.exists(path)|, 
      \lstinline|os.path.isdir(path)|}
    Return \texttt{True} if \texttt{path} exists / is a directory / is
    a regular file.
  \end{describe}
  
  \begin{describe}{\lstinline|os.path.basename(path)|, 
      \lstinline|os.path.dirname(path)|}
    Return the base name (the part after the last `\texttt{/}'
    character) or the directory name (the part before the last
    \texttt{/} character).
  \end{describe}
  
  \begin{describe}{\lstinline|os.path.abspath(path)|}
    Make \texttt{path} absolute (i.e., start with a \texttt{/}).
  \end{describe}
  
  \begin{references}
    \url{http://docs.python.org/library/os.path.html}
  \end{references}
\end{frame}


\begin{frame}[fragile]
  \begin{exercise}
    Write a Python program \texttt{rename.py} with the following
    command-line:
\begin{lstlisting}[language=sh]
python rename.py EXT1 EXT2 DIR [DIR ...]
\end{lstlisting}
    where:
    \begin{description}
    \item[ext1,ext2] Are file name extensions (without the leading
      dot), e.g., \texttt{jpg} and \texttt{jpeg}.
    \item[dir] Is directory path; possibly, many directories names can
      be given on the command-line.
    \end{description}

    The \texttt{rename.py} command should rename all files in
    directory DIR, that end with extension \texttt{ext1} to end with
    extension \texttt{ext2} instead.
  \end{exercise}
\end{frame}


\end{document}

%%% Local Variables: 
%%% mode: latex
%%% TeX-master: t
%%% End: 
