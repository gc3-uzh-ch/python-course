\documentclass[english,serif,mathserif,xcolor=pdftex,dvipsnames,table]{beamer}
\usetheme[informal]{s3it}
\usepackage{s3it}

\title[Warm-up]{%
  Warm-up exercise: \\
  analysis of ``response time'' data
}
\author[S3IT]{%
  S3IT: Services and Support for Science IT, \\
  University of Zurich
}
\date{June~23--24, 2014}

\begin{document}

% title frame
\maketitle


\begin{frame}[fragile]
  Suppose you are given a
  \href{https://raw.github.com/gc3-uzh-ch/python-course/master/rt.tsv}{file
    like this}:\footnote{%
  Many thanks to Franz Liem for providing the data.
}%
\begin{semiverbatim}
easy	0	2.161176607
medium	0	3.277300931
hard	0	3.662894132
hard	1	3.599218055
\end{semiverbatim}

\+
Each line is a ``trial'' and shows condition, if it was answered
correctly and the reaction time.
\end{frame}


\begin{frame}
  \begin{exercise}
    Read the
    \href{https://raw.github.com/gc3-uzh-ch/python-course/master/rt.tsv}{rt.tsv}
    file and output the average response time for condition
    \texttt{easy}.  Records relating to non-correct answers (i.e.,
    those for which the middle value is \texttt{0}) should be
    discarded.
  \end{exercise}

  \+
  \begin{exercise}
    Modify the solution to the previous exercise to be
    \emph{parametric}: i.e., it should be possible to specify what
    condition to look for (\texttt{easy}, \texttt{medium},
    \texttt{hard}), and if one wants records for which the answer is
    correct or not.
  \end{exercise}
\end{frame}


\begin{frame}
  \begin{exercise}
    \begin{enumerate}
    \item
      Write a function \texttt{parse\_data} that reads a given file (by default, file
      \href{https://raw.github.com/gc3-uzh-ch/python-course/master/rt.tsv}{\texttt{rt.tsv}})
      and returns a dictionary, mapping each condition name into a list
      of pairs \emph{(correct, reponse time)}.

    \item
      Write a function \texttt{analyze\_data} that takes three
      arguments: \texttt{data}, \texttt{condition}, and
      \texttt{max\_rt}, and returns the number of correct answers for
      the given condition whose response time is less than
      \texttt{max\_rt}.  The \texttt{data} argument is a dictionary
      like the one returned by the \texttt{parse\_data} function.
    \end{enumerate}
  \end{exercise}
\end{frame}

\end{document}

%%% Local Variables:
%%% mode: latex
%%% TeX-master: t
%%% End:
