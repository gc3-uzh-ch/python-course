\documentclass[english,serif,mathserif,xcolor=pdftex,dvipsnames,table]{beamer}
\usetheme[informal]{gc3}
\usepackage{gc3}

\title[Introduction]{%
  Typographical conventions used in the slides
}
\author[GC3]{%
  GC3: Grid Computing Competence Center, \\
  University of Zurich
}
\date{Mar.~19--20, 2014}

\begin{document}

% title frame
\maketitle


\begin{frame}
  \frametitle{Typographical conventions, I}

  The orange color is used for
  \href{http://www.gc3.uzh.ch/}{clickable
    links}; this should make it easy to download sample files, etc.

  \+
  Other \hl{colors} and \HL{backgrounds} are used for highlighting
  text in slides.
\end{frame}


\begin{frame}[fragile]
  \frametitle{Typographical conventions, II}

    \begin{columns}[t]
    \begin{column}{0.5\textwidth}
\begin{lstlisting}
# This is how Python
# code looks like

def hello(name):
  print ("Hello, " + name)
\end{lstlisting}
    \end{column}
    \begin{column}{0.5\textwidth}
      \raggedleft Commentary text appears on the right.
    \end{column}
  \end{columns}
\end{frame}


\begin{frame}[fragile]
  \frametitle{Typographical conventions, III}

    \begin{columns}[t]
    \begin{column}{0.5\textwidth}
\begin{lstlisting}
>>> ~\HL{print 2}~
2
\end{lstlisting}
    \end{column}
    \begin{column}{0.5\textwidth}
      \raggedleft
      This is an example of using the Python interactive shell.

      \+
      You should only type the highlighted part; the rest is
      provided by the Python interpreter.
    \end{column}
  \end{columns}
\end{frame}


\begin{frame}[fragile]
  \frametitle{Typographical conventions, IV}

    \begin{columns}[t]
    \begin{column}{0.5\textwidth}
\begin{lstlisting}
>>> print 2
~\HL{2}~
\end{lstlisting}
    \end{column}
    \begin{column}{0.5\textwidth}
      \raggedleft
      This is an example of using the Python interactive shell.

      \+
      The highlighted part is what the Python intepreter should
      reply to your command.
    \end{column}
  \end{columns}
\end{frame}


\begin{frame}[fragile]
  \frametitle{Typographical conventions, V}

    \begin{columns}[t]
    \begin{column}{0.5\textwidth}
\begin{lstlisting}
>>> print """A very
... ~\HL{long message."""}~
\end{lstlisting}
    \end{column}
    \begin{column}{0.5\textwidth}
      \raggedleft
      This is an example of using the Python interactive shell.

      \+
      The triple dots signal continuation lines,
      for when a Python command extends over multiple lines.
    \end{column}
  \end{columns}
\end{frame}


\end{document}

%%% Local Variables:
%%% mode: latex
%%% TeX-master: t
%%% End:
