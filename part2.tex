\documentclass[english,serif,mathserif,xcolor=pdftex,dvipsnames,table]{beamer}
\usepackage{gc3}

\title{%
  Python basics
}
\author[R. Murri]{%
  \textbf{Riccardo Murri} \\
  Grid Computing Competence Center, \\
  Organisch-Chemisches Institut, \\
  University of Zurich
}
\date{Sep.~28, 2012}

\begin{document}

% title frame
\maketitle


\begin{frame}[fragile]
  \frametitle{The REPL, I}
  Python is an \emph{interpreted} language.

  \+
  It also features an interactive ``shell''
  (\href{http://en.wikipedia.org/wiki/REPL}{REPL}) for evaluating
  expressions and statements immediately.

  \+
  The REPL is started by invoking the command \texttt{python} in a
  terminal window.
\begin{semiverbatim}\small
\$ \textbf{python}
Python 2.7.1+ (r271:86832, Apr 11 2011, 18:13:53) 
[GCC 4.5.2] on linux2
Type "help", "copyright", "credits" or "license" 
for more information.
>>> 
\end{semiverbatim}
\end{frame}

\begin{frame}[fragile]
  \frametitle{The REPL, II}
  Expressions can be entered at the REPL prompt \texttt{>>>}; they are
  evaluated and the result is printed:
\begin{semiverbatim}
>>> 2+2
4
\end{semiverbatim}

  \+
  A line can be continued onto the next by ending it with the
  character `\texttt{\textbackslash}'
\begin{semiverbatim}
>>> "hello" + \textbackslash
... " world!"
'hello world!'
\end{semiverbatim}
  The prompt changes to `\texttt{...}' on continuation lines.

  \+\scriptsize
  Reference:
  \url{http://docs.python.org/reference/lexical_analysis.html#line-structure}
\end{frame}


\begin{frame}[fragile]
  \frametitle{Expressions}

  Expressions are combinations of operations that manipulate values
  and yield some other values.  (Function calls are operations, too.)

  \+ 
  For instance, \texttt{2+2} is an expression, as are
  \texttt{abs(-2)}, \texttt{os.path.exists('/tmp')},
  \texttt{1 + (1.0/2) + 2**(-2)}

  \+ 
  When you type an expression at the Python prompt,
  it is evaluated and the resulting value is printed:
\begin{semiverbatim}
>>> abs(-2)
2
\end{semiverbatim}

  \+\scriptsize
  In formal terms, expressions are defined recursively:
  \begin{itemize}
  \item Every value is an expression;
  \item If $F$ is a $n$-ary operator and $E_1$, \ldots, $E_n$ are
    expressions, then $F(E_1, \ldots, E_n)$ is an expression.
  \end{itemize}
\end{frame}


\begin{frame}[fragile]
  \frametitle{Statements}
  Statements are commands that do not return a value.
\begin{semiverbatim}
>>> pass
{\small\it (No output)}
\end{semiverbatim}

  \+
  \textbf{The assignment operator is a statement:}
\begin{semiverbatim}
>>> a = 1
\end{semiverbatim}
  Similarly for the update operators \texttt{+=}, \texttt{-=}, etc.

  \+
  Other Python statements include definitions and
  control-flow constructs.  Everything else is an expression. 

  \+\scriptsize
  References:
  \url{http://lambda-the-ultimate.org/node/1044#comment-10878}
  \url{http://docs.python.org/reference/expressions.html}
  
\end{frame}


\begin{frame}[fragile]
  \frametitle{Statements, II}
  Well, almost: \texttt{print} is a \emph{statement} in Python 2:
\begin{semiverbatim}
>>> print 'a string'
a string
>>> print (2+2)
4
\end{semiverbatim}
  However, \texttt{print} is a \emph{function} in Python 3.

\end{frame}


\begin{frame}[fragile]
  \frametitle{Assignment, I}
  Assignment is done via the `\texttt{=}' statement:
\begin{semiverbatim}
>>> a = 1
>>> print a
1
\end{semiverbatim}
  \emph{Remember:} Assignement is a statement in Python, so no value is
  returned!

  \+
  \begin{question}
    Do you think the following will work?  If yes, what values are
    assigned to \texttt{a}, \texttt{b}, \texttt{c}?
    \begin{semiverbatim}
      >>> a = b = c = 1
    \end{semiverbatim}
    \only<2>{Yes, it works. It's a special exception in Python
      syntax. (Recall that Python strives for readability.)}
  \end{question}
\end{frame}


\begin{frame}[fragile]
  \frametitle{Assignment, II}
  Multiple assignments can be performed in one statement:
\begin{semiverbatim}
>>> a, b, c = 1, 2, 3
>>> print a
1
>>> print b, c
2 3
\end{semiverbatim}

  The multiple assignments are effected \emph{left-to-right, one after
  the other}.

\end{frame}


\begin{frame}
  \frametitle{Basic types}
  Basic object types in Python:
  \begin{description}
  \item[int] Integer numbers: \texttt{1}, \texttt{-2}, \ldots
  %   up to \texttt{9223372036854775807} (on a 64-bit machine)
  % \item[long] Integer numbers of arbitrary size; Python switches
  %   automatically from \texttt{int} to \texttt{long} when needed.
  \item[bool] The class of the two boolean constants \texttt{True}, \texttt{False}.
  \item[float] Double precision floating-point numbers, e.g.:
    \texttt{3.1415}, \texttt{-1e-3}.
  \item[str] Strings of byte-size characters.
  \item[unicode] Strings of UNICODE characters.
  \end{description}
\end{frame}

\begin{frame}[fragile]
  \frametitle{String literals, I}
  There are several ways to express string literals in Python.

  \+
  Single and double quotes can be used interchangeably:
\begin{semiverbatim}
>>> "a string" == 'a string'
True
\end{semiverbatim}

  \+
  You can use the single quotes inside double-quoted strings, and viceversa:
\begin{semiverbatim}
>>> a = "Isn't it ok?"
>>> b = '"Yes", he said.'
\end{semiverbatim}
\end{frame}


\begin{frame}[fragile]
  \frametitle{String literals, II}
  Multi-line strings are delimited by three quote characters.
\begin{lstlisting}
>>> a = """This is a string,
... that extends over more
... than one line.
... """
\end{lstlisting}
\end{frame}


\begin{frame}[fragile]
  \frametitle{Operators}
  All the usual unary and binary arithmetic and logical operators are
  defined in Python: \texttt{+}, \texttt{-}, \texttt{*}, \texttt{/},
  \texttt{**}~(exponentiation), \texttt{<<}, \texttt{>>}, etc.

  \+
  The comma operator (\texttt{,}) has a different meaning than in C/Java/etc.:
\begin{lstlisting}
>>> 1,2,3
(1, 2, 3)
\end{lstlisting}
  (It is the \texttt{tuple} constructor --- more on it later.)
\end{frame}

\begin{frame}[fragile]
  \frametitle{Operators, II}
  Some operators are defined for non-numeric types:
\begin{lstlisting}
>>> "U" + 'ZH'
'UZH'
\end{lstlisting}

  \+ 
  Some support operands of mixed type:
\begin{lstlisting}
>>> "a" * 2
'aa'
>>> 2 * "a"
'aa'
\end{lstlisting}

  \+
  Some do not:
\begin{lstlisting}[basicstyle=\footnotesize\ttfamily]
>>> "aaa" / 3
Traceback (most recent call last):
  File "<stdin>", line 1, in <module>
TypeError: unsupported operand type(s) for /: 'str' and 'int'
\end{lstlisting}
\end{frame}


\begin{frame}[fragile]
  \frametitle{Function calls}
  Functions are called by postfixing the function name with a
  parenthesized argument list.

  \+
\begin{lstlisting}
>>> abs(-4)
4
\end{lstlisting}

  \+
  In other words, \textbf{the \emph{postfix} \texttt{()} is the function call operator}.
\end{frame}


\begin{frame}[fragile,fragile]
  \frametitle{Functions, I}

  Type conversion is done via function calls:
  \begin{description}
  \item[str($x$)] Converts the argument $x$ to a string; for numbers,
    the base 10 representation is used.
  \item[int($x$)] Converts its argument $x$ (a number or a string) to an integer;
    decimal digits are truncated.
  \item[float($x$)] Converts its argument $x$ (a number or a string) to a
    floating-point number.
  \end{description}

  \+ 
  Example:
\begin{semiverbatim}
>>> str(2)
'2'
>>> float(2)
2.0
\end{semiverbatim}
\end{frame}


\begin{frame}[fragile]
  \frametitle{Functions, II}
  Some functions can take a variable number of arguments:
  \begin{description}
    \item[sum($x_0$, \ldots, $x_n$)] Return $x_0 + \cdots + x_n$.
    \item[max($x_0$, \ldots, $x_n$)] Return the maximum of the set $\{ x_0, \ldots, x_n \}$
    \item[min($x_0$, \ldots, $x_n$)] Return the minimum of the set $\{ x_0, \ldots, x_n \}$
  \end{description}

  \+ 
  Examples:
\begin{semiverbatim}
>>> sum(1,2,3)
6
>>> max(1,2)
2
\end{semiverbatim}
\end{frame}


\begin{frame}[fragile]
  \frametitle{The most important function of all}
  \begin{description}
  \item[help(\texttt{fn})] Display help on the function named \texttt{fn}
  \end{description}
  
  \+
  \begin{question}
    What happens if you type these at the prompt?
    \begin{columns}
      \begin{column}{0.45\textwidth}
        \texttt{>>> help(abs)}
      \end{column}
      \begin{column}{0.45\textwidth}
        \texttt{>>> help(int)}
      \end{column}
    \end{columns}
  \end{question}
\end{frame}

\begin{frame}
  \frametitle{The most important function of all, II}

  When called without any argument, \hl{help()} starts an interactive
  help prompt.

  \+
  \begin{semiverbatim}\tiny
>>> help()

Welcome to Python 2.7!  This is the online help utility.

If this is your first time using Python, you should definitely check out
the tutorial on the Internet at http://docs.python.org/tutorial/.

Enter the name of any module, keyword, or topic to get help on writing
Python programs and using Python modules.  To quit this help utility and
return to the interpreter, just type "quit".

To get a list of available modules, keywords, or topics, type "modules",
"keywords", or "topics".  Each module also comes with a one-line summary
of what it does; to list the modules whose summaries contain a given word
such as "spam", type "modules spam".

help> 
\end{semiverbatim}

  \+
  To return to the normal prompt, type \texttt{quit}

  \+ 
  \hl{help(\texttt{'topic'})} has the same effect as typing
  \texttt{topic} at the interactive help prompt.
\end{frame}

\begin{frame}[fragile]
  \frametitle{How to define new functions}
  \begin{columns}[t]
    \begin{column}{0.5\textwidth}
\begin{lstlisting}
@\HL{\textbf{def} hello(name):}@
  """
  A friendly function.
  """
  print ("Hello, " + name + "!")

# the customary greeting
hello("world")
\end{lstlisting}
    \end{column}
    \begin{column}{0.5\textwidth}
      \raggedleft 
      The \textbf{def} statement starts a function definition.
    \end{column}
  \end{columns}
\end{frame}

\begin{frame}[fragile]
  \begin{columns}[t]
    \begin{column}{0.5\textwidth}
\begin{lstlisting}
def hello(name):
  """
  A friendly function.
  """
  @\HL{print ("Hello, " + name + "!")}@

# the customary greeting
hello("world")
\end{lstlisting}
    \end{column}
    \begin{column}{0.5\textwidth}
      \raggedleft 
      \textbf{Indentation is significant in Python}: it is used to delimit
      blocks of code, like `\texttt{\{}' and `\texttt{\}}' in Java and C.
    \end{column}
  \end{columns}
\end{frame}

\begin{frame}[fragile]
  \begin{columns}[t]
    \begin{column}{0.5\textwidth}
\begin{lstlisting}
def hello(name):
  """
  A friendly function.
  """
  print ("Hello, " + name + "!")

@\HL{\it\tt\#\ the customary greeting}@
hello("world")
\end{lstlisting}
    \end{column}
    \begin{column}{0.5\textwidth}
      \raggedleft 
      (This is a comment. It is ignored by Python, just like blank lines.)
    \end{column}
  \end{columns}
\end{frame}

\begin{frame}[fragile]
  \begin{columns}[t]
    \begin{column}{0.5\textwidth}
\begin{lstlisting}
def hello(name):
  """
  A friendly function.
  """
  print ("Hello, " + name + "!")

# the customary greeting
@\HL{hello("world")}@
\end{lstlisting}
    \end{column}
    \begin{column}{0.5\textwidth}
      \raggedleft 
      This calls the function just defined.
    \end{column}
  \end{columns}
\end{frame}

\begin{frame}[fragile]
  \begin{columns}[t]
    \begin{column}{0.5\textwidth}
\begin{lstlisting}
def hello(name):
  @\HL{"""}@
  @\HL{A friendly function.}@
  @\HL{"""}@
  print ("Hello, " + name + "!")

# the customary greeting
hello("world")
\end{lstlisting}
    \end{column}
    \begin{column}{0.5\textwidth}
      \raggedleft 
      What is this? The answer in the next exercise!
    \end{column}
  \end{columns}
\end{frame}

\begin{frame}
  \begin{exercise}
    Type and run the code on the previous page at the interactive
    prompt. (Type indentation spaces, too!)
    
    What does \texttt{help(hello)} print?  
    What's the result of evaluating the function \texttt{hello("world")}?
  \end{exercise}

  \+
  \begin{exercise}
    Type the same code in a file named \texttt{hello.py}, then type
    \texttt{import hello} at the interactive prompt.
    What happens?
  \end{exercise}  
\end{frame}


\begin{frame}[fragile]
  \frametitle{Modules}
  The \texttt{import} statement reads a \texttt{.py} file and makes
  its contents available to the current program.
\begin{lstlisting}
>>> import hello
>>> hello.hello("Bob")
Hello, Bob!
\end{lstlisting}

  \+
  To import definitions into the current namespace, use the
  `\texttt{from $x$ import $y$}' form:
\begin{lstlisting}
>>> from fractions import Fraction
\end{lstlisting}

  \+
  \textbf{Modules are only read once}, no matter how many times an
  \texttt{import} statement is issued.
\end{frame}

\begin{frame}[fragile]
  Conditional execution uses the \texttt{if} statement:
\begin{lstlisting}
if @\it expr@:
  # indented block
elif @\it other-expr@:
  # indented block
else:
  # executed if none of the above matched
\end{lstlisting}

  \+ The \texttt{elif} can be repeated, with different conditions, or
  left out entirely.

  \+ 
  Also the \texttt{else} clause is optional.

  \+
  \begin{question}
    Where's the `end if'?

    \only<2>{There's no `end if': indentation delimits blocks!}
  \end{question}
\end{frame}


\begin{frame}[fragile]
  Conditional looping uses the \texttt{while} statement:
\begin{lstlisting}
while @\it expr@:
  # indented block
else:
  # executed at natural end of the loop
\end{lstlisting}

  \+
  To break out of a \texttt{while} loop, use the \texttt{break}
  statement. 

  \+
  If a loop is exited via a \texttt{break} statement, the
  \texttt{else} clause is \emph{not} executed.
\end{frame}


\end{document}


%%% Local Variables: 
%%% mode: latex
%%% TeX-master: t
%%% End: 
